% Suppress some acceptable compilation warnings
\RequirePackage[save,showerrors]{silence}
\WarningFilter{multicol}
  {May not work with the twocolumn option} % I'm not really using the multicol package
\WarningFilter{latex}
  {Marginpar on page} % Margin stuff moved to not overlap with other stuff, this is fine

% llncs class documentation:
% https://ctan.tetaneutral.net/macros/latex/contrib/llncs/llncsdoc.pdf
\documentclass[twocolumn,dvipsnames]{llncs}

% language settings
\usepackage[main=french,english]{babel}
\newcommand{\en}[1]{\foreignlanguage{english}{\todo{fix weird hspace}\emph{#1}}}
\usepackage{csquotes}
\MakeOuterQuote{"}

% bibliography settings
\usepackage[hyperref=true,backref=true]{biblatex}
\DefineBibliographyExtras{french}{\restorecommand\mkbibnamefamily}
\addbibresource{../references.bib}

\usepackage{todonotes}

% links setup
\usepackage[nobiblatex]{xurl}
\usepackage{hyperref}
\hypersetup{colorlinks=true}

\title{L'Identification des Projets de Logiciel Libre Accessibles aux Nouveaux Contributeurs}
\author{%
    \todo[inline]{TODO: anonymiser les auteurs pour la soumission initiale}
    Benoît Crespin\inst{1}\orcidID{0000-0002-9105-0243}%
    \and%
    Paul Hervot\inst{1}%
}
\institute{Université de Strasbourg}

\begin{document}
    \maketitle

    \begin{abstract}
        \todo{TODO abstract}
        \keywords{
            Logiciel libre \and Analyse automatique de dépôts logiciels \and
            Barrières d'entrée du logiciel libre \and Nouveaux contributeurs
        }
    \end{abstract}

    \section{Introduction}
    \todo{TODO : sans doute à compacter et enlever les sous-sections}
    \subsection{Le logiciel libre}
    Au cours des quelques décennies précédentes, le numérique s'est développé
    pour intégrer et assister la plupart des domaines de recherche, des secteurs
    industriels, ainsi que des aspects de nos vies. Son champ est aujourd'hui
    très vaste et connaît en son sein de nombreuses dynamiques différentes ayant
    un impact sur ses domaines d'applications, parmi lesquelles celle du
    logiciel libre.

    En essence, les logiciels, applications, services numériques et autres
    programmes sont des ensembles d'instructions compréhensibles par un
    ordinateur et lui indiquant comment utiliser ses ressources afin de réaliser
    un certain nombre de tâches. Pour être utilisables par un ordinateur, ces
    instructions doivent être formulées en "code machine", un langage souvent
    représenté en binaire et extrêmement difficile à manipuler par les humains,
    même les plus spécialisés, aussi bien en ce qui concerne son écriture que sa
    compréhension. C'est pourquoi lors de la création d'un logiciel, les
    développeurs décrivent d'abord le comportement souhaité dans un autre
    langage, textuel et raisonnablement maîtrisable pour un être humain
    spécialisé, que l'on appelle un "langage de programmation". Cette
    description textuelle du logiciel est ce que l'on appelle son "code source",
    il est ensuite traduit en code machine par un autre programme (généralement
    appelé un "compilateur") avant d'être livré aux personnes souhaitant
    utiliser le logiciel.

    Pour utiliser un logiciel, nous n'avons donc besoin que de son code machine,
    mais c'est seulement en lisant son code \emph{source} que l'on peut
    vraisemblablement comprendre son fonctionnement, corriger ses éventuelles
    erreurs, l'adapter à d'autres besoins, vérifier la présence de comportements
    malveillants, etc. La question de rendre publiquement disponible le code
    source d'un logiciel est un enjeu faisant souvent intervenir plusieurs
    intérêts divergents, allant de la transparence du fonctionnement de certains
    services et leur gestion démocratique au secret industriel.

    L'expression "logiciel libre" désigne en français un logiciel disponible
    publiquement et gratuitement dont le code source est lui aussi disponible
    publiquement et gratuitement, ainsi que sa modification et sa redistribution
    par toutes et tous. Cette expression désigne aussi et surtout un mouvement
    visant à développer la part de logiciel libre dans le numérique, un
    mouvement dans lequel se développe toute une culture, des pratiques, des
    fondations et des événements internationaux. L'expression équivalente
    anglaise, "\en{Free and Open Source Software}" (FOSS) est souvent utilisée
    dans les références citées dans ce mémoire.

    \subsection{Les projets de logiciel libre dans l'enseignement de l'informatique}
    La pédagogie de projet est une méthode d'enseignement qui consiste à inviter
    les apprenants à appliquer leurs connaissances et à en acquérir de nouvelles
    au travers d'un projet plus ou moins concret et imitant plus ou moins
    fidèlement une situation réelle. Réalisé soit individuellement soit en
    groupe, les projets aboutissent généralement à une production évaluée par
    l'enseignant, parfois au travers d'une présentation donnée par les
    apprenants. L'efficacité de cette méthode a fait l'objet de nombreuses
    recherches au cours des années précédentes. Plusieurs méta-analyses
    concluent que cette méthode a des effets mesurables, positifs et importants
    sur les résultats académiques des apprenants en sciences sociales et en
    sciences naturelles \cite{pbl-2019, pbl-2018}.

    Les projets proposés aux étudiants sont le plus souvent factices, imaginés
    par les enseignants dans le seul but de servir d'exercice. Utiliser au
    contraire des projets réels, qui ne cessent pas d'exister après la fin de la
    séquence pédagogique, sur lesquels faire travailler les étudiants donne des
    résultats encore meilleurs, mais est aussi plus difficile et incertain à
    encadrer pour les enseignants \cite{real-pbl-2010, real-pbl-2004}. Il est
    notamment très difficile pour l'enseignant de prévoir à l'avance les
    problèmes que les apprenants risquent de rencontrer, c'est pourquoi d'autres
    efforts dans cette direction ont tenté un compromis, consistant à créer des
    projets de logiciel libre aussi proches des situations réelles que possible,
    mais dédiés à l'éducation \cite{oss-edu-2008}. Les projets en question ne
    sont cependant plus accessibles aujourd'hui, une explication possible à leur
    disparition étant leur portée réduite à l'éducation et l'absence d'une
    communauté persistante autour d'eux.

    Bien qu'elles soient plutôt rares, quelques initiatives existent aussi pour
    enseigner spécifiquement les pratiques du logiciel libre dans l'éducation
    supérieur et la formation tout au long de la vie. Certaines sont des
    initiatives venant des communautés du logiciel libre, comme le
    \en{Professional Certificate in Open Source Software Development, Linux and
    Git}%
    \footnote{\url{https://www.edx.org/professional-certificate/linuxfoundationx-open-source-software-development-linux-and-git}}
    de la \en{Linux Foundation}, et en particulier la première séquence :
    \en{Open Source Software Development: Linux for Developers}%
    \footnote{\url{https://www.edx.org/course/open-source-software-development-linux-for-developers}} ;
    ou les séminaires de l'\en{open source
    initiative}\footnote{\url{https://opensource.org/osi-open-source-education}}.
    Enfin, certaines universités proposent des cursus en informatique ayant des
    éléments visant spécifiquement les pratiques du logiciel libre, c'est le cas
    notamment de l'Université de l'État de Caroline du Nord aux États-Unis, qui
    a expérimenté plusieurs façons d'inclure la contribution aux projets de
    logiciel libre dans leurs cursus d'informatique \cite{oss-edu-2008,
    oss-edu-2007} ; mais aussi de l'université de Calais qui propose
    actuellement un "Master Informatique - Ingénierie du logiciel libre"%
    \footnote{\url{https://www.univ-littoral.fr/formation/offre-de-formation/masters/master-informatique-ingenierie-du-logiciel-libre/}},
    sous la forme d'une formation en alternance.

    \section{Revue de littérature}

    \section{Méthodologie}

    \section{Résultats et discussion}

    \section{Conclusion}

    \printbibliography[heading=bibintoc]
\end{document}
