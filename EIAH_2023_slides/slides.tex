% Suppress some compilation warnings
\RequirePackage[save,showerrors]{silence}
\WarningFilter{biblatex}
  {The starred command '\DeclareDelimAlias*' is} % APA package is using this deprecated starred version
\WarningFilter{transparent}
  {Loading aborted} % Used by the svg package

\documentclass[usenames,dvipsnames,10pt]{beamer}

% languages typesetting rules
\usepackage[main=french,english]{babel}

% links configuration
\usepackage{hyperref} % commands such as \href, \url, etc.
\hypersetup{
    colorlinks=true,
    linkcolor=,
    urlcolor=NavyBlue,
    citecolor=Green
}

% bibliography configuration
\usepackage[style=lncs,backend=biber,backref=true]{biblatex}

\usepackage{csquotes} % biblatex extensions for non-english languages
\usepackage{caption} % more control over captions
\usepackage{subcaption} % for subfigures
\usepackage{ccicons} % icons for Creative-Commons licenses
\usepackage{hologo} % tex and co. logos
\usepackage{siunitx} % for displaying large numbers
\usepackage[inkscapepath=./build/svg-inkscape/]{svg}

\graphicspath{{images/} {../images}}

% Beamer theme configuration
\usetheme{metropolis}
\metroset{sectionpage=progressbar, progressbar=frametitle}
\setbeamertemplate{caption}[numbered] % Show figure number when using `\caption` instead of `\caption*`
\setbeamertemplate{footline}[page number] % Show the current page number in the bottom line of each slide
% \setbeameroption{show only notes}

% Bibliography configuration
\bibliography{../references.bib}

% Meta information about this presentation
\title{L'identification des projets de logiciel libre accessibles aux nouveaux contributeurs}
\titlegraphic{\includegraphics[width=0.25\textwidth]{EIAH_2023}}
\institute[]{\href{https://creativecommons.org/licenses/by-sa/4.0/}{\ccbysa}}
\author{%
    Paul Hervot\inst{1}%
    \and%
    Benoît Crespin\inst{2}%
}
\institute{
    \textsuperscript{1} Laboratoire de Recherche de L’EPITA (LRE), 14-16 rue Voltaire, \\94270 Le Kremlin-Bicêtre, France
    \and
    \textsuperscript{2} Université de Limoges, XLIM/ASALI, UMR CNRS 7252, France
}
\date{15 juin 2023}

\newcommand{\mycite}[1]{%
    \citeauthor{#1} \citeyear{#1} \cite{#1}%
}

\begin{document}

\frame{\titlepage{}}

\section{Contexte}

\begin{frame}{Mémoire de Master}
    Formation d'ingénieur, enseignant à EPITA.

    Mémoire supervisé par Benoît Crespin.
    
    \note[item]{Très rapide, juste histoire de dire que je ne suis pas issu
    moi-même du milieu EIAH}
\end{frame}

\begin{frame}[fragile]{Contributions au logiciel libre}
    Nombreuses barrières d'accès, souvent discriminantes :
    \mycite{barriers-2018}, \mycite{newcomers-accessibility-2016},
    \mycite{newcomers-onboarding-2018}.

    \bigskip\bigskip

    Souvent liées aux propriétés des projets $\implies$ peut-on identifier
    lesquels sont accessibles ?

    \note[item]{%
        Même des projets populaires comme Apache Hadoop ont un très
        faible taux de rétention des nouveaux contributeurs, $82\%$ d'après
        Mendez et al. Les barrières sont très variées et sont souvent
        actionables par le projet lui-même : elles concernent la documentation
        disponible, la complexité des outils nécessaires et du setup à mettre en
        place pour commencer à travailler, ainsi que le comportement des
        mainteneurs.
    }
\end{frame}

\begin{frame}[fragile]{Inspiration}
    \mycite{signals-2019} : quels signaux utilisent les aspirants contributeurs.

    \begin{minipage}{.49\textwidth}
        \begin{figure}
            \includegraphics[width=\textwidth]{qiu_overview}
        \end{figure}
    \end{minipage}
    \begin{minipage}{.49\textwidth}
        \begin{figure}
            \includegraphics[width=\textwidth]{qiu_regressions}
        \end{figure}
    \end{minipage}

    \note[item]{%
        Je me suis beaucoup inspiré de cet article de Huilian Sophie Qiu et al.
        qui identifie certains signaux que les aspirants contributeurs regardent
        dans un projet pour choisir auquel contribuer, puis mesure la
        correlation qu'il y a entre ces signaux et la présence de tentative de
        contributions.

        Ils trouvent en particulier qu'un grand nombre de \emph{commits} récents
        est lié à une meilleur attractivité, contrairement à la présence
        d'instructions de contribution. Leur étude mesure aussi de façon très
        intéressante l'impact de la politesse des mainteneurs, ce que mon
        approche ne me permettra pas de faire.

        Mon but est de vérifier si ces signaux marchent aussi pour identifier
        les projets \emph{accessibles} (les contributeurs arrivent à contribuer,
        leur code est merged dans le projet) et non seulement attractifs (les
        gens \emph{essayent} de contribuer).
    }
\end{frame}

\section{Méthodologie}

\begin{frame}[fragile]{Pour dépasser les limites de GitHub}
    \mycite{mining-github-2014}, \mycite{penumbra-oss-2022} : limites du minage
    mono-plateforme.

    \mycite{barriers-2018} : limites des études se concentrant sur quelques gros
    projets uniquement.

    \bigskip

    $\implies$ Besoin d'une stratégie de minage de données plus représentative
    et générique.

    \begin{figure}
        \includesvg[width=0.5\textwidth]{softwareheritage}
    \end{figure}

    \note[item]{%
        Autre objectif : dépasser les limites de l'analyse de données issues de
        GitHub (beaucoup de projets "personnels", beaucoup de PR non marquées
        comme merged alors qu'elles l'ont été à la main, etc.).
    }
\end{frame}

\begin{frame}{Pertinence pour la recherche}
    \mycite{swh-2019}, \mycite{swh-seirl}.

    \begin{itemize}
        \item $+$\num{239000000} projets
        \item $+$\num{3000000000} \emph{commits} uniques
        \item Parcours régulièrement Bitbucket, GitHub, gitlab.com, CRAN, Maven,
            gnu.org, NPM, pypi, sourceforge, \ldots
        \item google code, autres GitLabs, autres origines à la demande
    \end{itemize}

    \note[item]{%
        Software Heritage maintien la plus grande archive logicielle publique au
        monde avec l'objectif spécifique de la rendre exploitable à des fins de
        recherche. L'archive contient des projets issus de nombreuses sources
        différentes (GitHub restant la principale), y compris venant de
        logiciels de versionnement différents, et tous représentés sous un seul
        grand graphe de \emph{commits} dédupliqués.
    }
\end{frame}

\begin{frame}{Un grand \emph{Merkle DAG}}
    \begin{minipage}{.46\textwidth}
        \begin{figure}
            \includegraphics[width=\textwidth]{swh-graph}
        \end{figure}

        \mycite{swh-2019}
    \end{minipage}
    \begin{minipage}{.52\textwidth}
        Exclus de l'étude : \begin{itemize}
            \item les \emph{forks}
            \item les projets inactifs
            \item les projets non colaboratifs
        \end{itemize}

        \bigskip

        Données de recherche : \begin{itemize}
            \item nombre de nouveaux contributeurs (variable expliquée)
            \item nombre de contributeurs uniques récents
            \item nombre de \emph{commits} récents
            \item présence d'instructions de contribution
        \end{itemize}
    \end{minipage}
    
    \note[item]{%
        C'est un Merkle DAG, donc un graphe orienté sans cycle dont tous les
        nœuds sont identifiés par un hash unique, ce qui permet de dédupliquer.

        La vue ici est pour une "origine" (URL à partir de laquelle un archivage
        peut être effectué). Les infos qui nous intéressent sont les snapshots
        (date d'archivage), les révisions et, plus tard, les contenus des
        fichiers.

        Je commence donc par parcourir *tous* les nœuds du graphe, à chaque fois
        que je trouve une origine je lance une découverte de composante connexe
        en considérant aussi les arcs inverses des révisions (c'est à dire les
        clusters de fork) ce qui me permet de sélectionner une origine
        "représentante" des fork (celle la plus éloignée d'un \emph{commit}
        initial) et d'éliminer toutes les autres de mon parcours.

        Ensuite, pour chacun des projets retenus, je lance un parcours largeur à
        partir de ce qui ressemble le plus à une branche principale afin de
        récolter les données de recherche. Tout est prélevé uniquement au sein
        d'une période de référence bien définie (et le caractère "récent" est
        une autre période bien définie précédant la période de référence).
    }
\end{frame}

\begin{frame}{Un \textbf{GRAND} \emph{Merkle DAG}}
    \centering

    Essentiellement trois parcours largeur
    \pause

    Entièrement en RAM (7 TiB), 48 cœurs
    \pause

    Structures et algorithmes optimisés, peu de synchronisation
    \pause
    \bigskip
    \bigskip

    {\huge Trois jours}\\{\footnotesize (et un jour de plus pour les READMEs)}

    \note[item]{%
        Software Heritage a été assez gentil pour faire tourner mon code sur
        leur serveur capable de faire tenir tout le graphe en RAM, pendant deux
        semaines où personne d'autre ne s'en servait j'ai pu utiliser les 48
        cœurs et paralléliser à fond.

        Heureusement y'a sous-graphes de taille raisonnable dont j'ai pu
        me servir pour tester mon code de collecte chez moi)
    }
\end{frame}

\section{Résultats}

\begin{frame}{Instructions de contribution}
    \begin{figure}
        \includegraphics[width=.5\textwidth]{../experiment/data_analysis/hasContrib_meanNewContributorCount}
        \caption{%
            Moyenne du nombre de nouveaux contributeurs en fonction de la
            présence d'instructions de contribution
        }
    \end{figure}

    Test MWW : $\rho \approx 0.57$, $p \approx 0$

    \note[item]{%
        C'est très léger, mais d'après le test de Wilcoxon-Mann-Whitney il y a
        effectivement un lien entre la présence d'instructions de contribution
        d'un projet et son accessibilité pour les nouveaux contributeurs
    }
\end{frame}

\begin{frame}{Nombre de contributeurs uniques récents}
    \begin{figure}
        \includegraphics[width=.5\textwidth]{../experiment/data_analysis/recentContributorCountRegression_linearScale}
        \caption{%
            Nombre de nouveaux contributeurs en fonction du nombre de
            contributeurs uniques récents
        }
    \end{figure}

    Régression GLS : $R^2 \approx 0.45$

    \note[item]{%
        Là aussi c'est léger mais pareil, on voit un lien avec un coefficient de
        détermination de $45\%$. La difficulté c'est que la distribution des
        données n'est pas du tout normale, d'où l'utilisation d'un modèle
        utilisant la méthode des moindres carrés.
    }
\end{frame}

\begin{frame}{Nombre de \emph{commits} récents}
    \begin{figure}
        \includegraphics[width=.5\textwidth]{../experiment/data_analysis/recentCommitCountRegression_linearScale}
        \caption{%
            Nombre de nouveaux contributeurs en fonction du nombre de
            \emph{commits} récents
        }
    \end{figure}

    Régression GLS : $R^2 \approx 0.10$

    \note[item]{%
        Là c'est encore plus léger et le coefficient de détermination est à
        seulement $10\%$, c'est donc le seul lien que je ne retiens pas comme
        significatif.
    }
\end{frame}

\begin{frame}{Conclusion}
    % TODO
\end{frame}

\begin{frame}{Contact}
    \begin{itemize}
        \item Contact : \href{mailto:paul.hervot@epita.fr}{paul.hervot@epita.fr}
        \item Dépôt GitHub :
            \href{https://github.com/Dettorer/synva-dissertation}{Dettorer/synva-dissertation}
        \item \emph{Replication package} :
            \href{https://zenodo.org/record/7023495}{zenodo.org/record/7023495}
    \end{itemize}

    \note[item]{%
        J'ai essayé de rendre le code le plus lisible et réutilisable possible.
    }
\end{frame}

\section{Bibliographie}

\begin{frame}[allowframebreaks]{Bibliographie}
    \printbibliography{}
\end{frame}

\end{document}
