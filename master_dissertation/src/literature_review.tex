\chapter{Revue de littérature}

\section{L'intérêt professionnel et personnel dans la contribution aux projets de logiciel libre}

L'industrie informatique s'intéresse de plus en plus au logiciel libre et à son inclusion ou ses interactions
avec ses activités, au point de ressentir un besoin de plus en plus important en terme de formation sur ce
sujet, ainsi qu'une volonté à participer aux projets utilisés au sein des entreprises
\sideparencite{strategies-2012}. Même quand les logiciels libres ne sont pas directement impliqués dans les
activités d'une entreprise, d'autres auteurs avancent que les personnes produisant des contributions à de tels
projets envoient un signal positif à leurs pairs et augmentent leurs chances de succès professionnel dans
l'industrie informatique \sideparencite[][p.~218]{oss-economics-2002}.

\textcite{strategies-2012} identifient cinq grandes étapes dans les interactions que les entreprises ont
généralement avec les logiciels libres :
\begin{description}
    \item[l'identification] -- évaluer la qualité d'un logiciel libre, ses garanties et les éventuels brevets
        utilisés ;
    \item[l'adoption] -- utilisation du logiciel ;
    \item[la conformité] -- examen des licence attachées au logiciel, de leur compatibilité avec les activités
        de l'entreprise, et des contraintes qu'elles apportent dans les activités futures ;
    \item[la contribution] -- partage avec la communauté.
\end{description}

Les motivations poussant certaines personnes à contribuer au logiciel libre sont aussi personnelles.
\sidetextcite{os-personal-interests-2008} distinguent l'importance de ces motivations en fonction du contexte
de la contribution. Lorsqu'il s'agit de contribuer au code source d'un projet de logiciel libre, les
contributeurs semblent principalement poussés par un intérêt de développement personnel (apprendre à utiliser
de nouvelles technologies ou consolider son expertise par la pratique), puis par une volonté altruiste d'aider
le projet, puis en dernier lieu par la publicité des contributions, dont ils se servent pour se construire une
réputation. Lorsqu'il s'agit de contribuer à un autre type de contenu que le code source en revanche (les
articles de l'encyclopédie Wikipédia par exemple), c'est la volonté altruiste qui semble être le moteur
principal amenant à une contribution, puis le développement personnel et, là encore, l'établissement d'une
réputation.

\section{Les nouveaux contributeurs dans le logiciel libre}

\subsection{Les barrières d'entrée}

\sidetextcite{barriers-2018} notent la difficulté des nouveaux volontaires à rejoindre une communauté de
logiciel libre, citant comme exemple extrême le projet "Apache Hadoop" voyant 82\% de ses nouveaux volontaires
quitter le projet après leur première contribution \sideparencite{hadoop-dropout-2013}. Les études qu'ils
citent mentionnent deux approches différentes de la résolution de problèmes observées dans les projets
informatique : l'une consiste à d'abord rassembler le plus d'informations possibles sur le problème avant de
tenter en un deuxième temps de le résoudre (approche statistiquement plus commune chez les femmes que chez les
hommes), l'autre consiste à agir sur la première information ou piste prometteuse trouvée, quitte à revenir en
arrière et chercher de nouvelles informations si elles n'étaient pas concluantes (approche statistiquement
plus commune chez les hommes que chez les femmes)
\sideparencite{gender-information-processing-1995,gender-information-processing-2015}. De façon plus précise
concernant les logiciels, on observe deux façons d'apprendre les fonctionnalités d'un logiciel statistiquement
préférées de façon inégales selon le genre : l'une, statistiquement plus représentée chez les femmes, consiste
à les apprendre méthodiquement en suivant des processus d'apprentissage clairs. L'autre, statistiquement plus
représentée chez les hommes, consiste à expérimenter à la manière d'un jeu avec ces fonctionnalités
\sideparencite{gender-programming-2010,gender-programming-2006}. Une des tendances trouvées par
\textcite[p.~1008]{barriers-2018} est que la majorité ($58\%$) des barrières rencontrées dans leur étude sont
de nature sociale ("\en{community-oriented}") et non techniques. Concernant les aspects plus techniques, il
semble que les outils et la documentation représentent la majorité des barrières rencontrées, alors même que
ces éléments ont spécifiquement pour objectif d'aider les nouvelles contributions. C'est un type de barrière
qui semble amplifié lorsque les méthodes d'apprentissage et de traitement de l'information de la personne
consistent à rassembler le plus d'information possible avant de commencer à agir. Ces méthodes étant
sur-représentées chez les femmes, ces barrières deviennent discriminantes selon le genre, et peuvent
potentiellement participer à expliquer la sous-représentation des femmes dans les communautés de logiciel
libre.

Plus précisément en ce qui concerne les contributions, selon \sidetextcite{hadoop-dropout-2013}, les facteurs
susceptibles de provoquer un abandon des nouveaux contributeurs semblent être les réponses inadéquates
proposées à leurs questions, notamment lorsqu'un autre nouveau contributeur répond à la place d'un membre
expérimenté du projet.

Il a été théorisé et souvent observé de façon générale que la diversité de genre, d'origines et d'ancienneté
au sein d'une équipe augmente sa productivité \sideparencite[][voir par exemple][]{diversity-2007}. C'est ce
que \sidetextcite{diversity-2015} ont pu confirmer dans le cas précis des projets de logiciel libre hébergés
sur \gls{github}. Ils mettent cette observation en perspective avec celle de la proportion de femmes dans ce
type d'équipe, en très forte minorité, et concluent en suggérant que des efforts supplémentaires d'inclusivité
et de réduction des barrières d'entrée pour ces population en particulier permettraient probablement
d'augmenter la valeur globale de ce type de projets.

La littérature sur le sujet identifie un grand nombre de barrières d'entrée que rencontrent les nouveaux
arrivants dans un projet de logiciel libre, les plus importantes semblent être :

\begin{itemize}
    \item le manque d'interaction sociale avec les membres du projet
        \sideparencite[-1cm]{barriers-meta-2015,social-barriers-2015} ;
    \item le manque de réponse (dans un temps raisonnable) à leurs requêtes \parencite{barriers-meta-2015} ;
    \item les connaissances techniques initiales \parencite{barriers-meta-2015} ;
    \item l'identification d'une tâche par laquelle commencer \sideparencite{first-task-2015} ;
    \item la mise en place de l'environnement propre au projet abordé et permettant de faire une première
        contribution \sideparencite{newcomers-accessibility-2016}.
\end{itemize}

\subsection{Mesure de l'accessibilité d'un projet pour les nouveaux contributeurs}
\label{sec:accessibility-measure}

Ces modèles de barrières d'entrée permettent d'identifier les freins que rencontrent les nouveaux
contributeurs dans un projet de logiciel libre. Un autre point de vue sur la question consiste à mesurer
\emph{a posteriori} la capacité d'un projet à accueillir les nouveaux contributeurs et à leur permettre de
mener leurs contributions à terme.

Une première approche à cela, qualitative, est de proposer à un ensemble maîtrisé d'étudiants d'essayer de
contribuer à un projet et de leur soumettre un questionnaire avant et/ou après l'expérience afin de recueillir
leur ressenti concernant les problèmes qu'ils pensent rencontrer et ceux qu'ils ont effectivement rencontré,
puis d'éventuellement compléter avec des entretiens individuels afin d'explorer les mécanismes liés à ces
problèmes \parencites{newcomers-accessibility-2016}{newcomers-onboarding-2018}[voir
aussi][]{newcomers-adaptation-2005}.

Une autre approche, peut être plus adaptée aux recherches quantitatives, consiste à compter automatiquement le
nombre de contributeurs accumulés depuis le début de la vie du projet et jusqu'à différents points dans le
temps, afin d'étudier la progression de ce nombre. C'est ce que font par exemple
\sidetextcite{contributor-count-2006}, mais sans préciser la méthode de comptage plus en détail que "en
utilisant plusieurs outils d'édition de texte" (p. 116, traduit).

\sidetextcite{signals-2019} traitent d'une question similaire en essayant de déterminer quels sont les signaux
que les potentiels nouveaux contributeurs regardent (et comment) pour choisir un projet de logiciel libre
auquel contribuer. Ils s'intéressent pour cela spécifiquement aux projets disponibles publiquement sur
\gls{github}, ce qui leur permet, après une approche exploratoire pour dégager des hypothèses, d'étudier
empiriquement l'effet des signaux que la plateforme met en évidence.

\textcite{signals-2019} suggèrent qu'une mesure possible de l'accessibilité pour les nouveaux contributeurs
("\en{newcomers openness}" dans l'article) est le pourcentage de \glspl{pull request} créées par des
contributeurs externes. Ils définissent les "contributeurs externes" par opposition aux contributeurs
principaux (les "\en{core contributors}") qu'ils identifient, eux, comme étant les personnes auteurs de plus
de 5\% des \glspl{commit} du projet. En réalité cependant, leur étude quantitative simplifie la mesure en ne
retenant que la présence ou non de \gls{pull request} venant de nouveaux contributeurs (p.~16), ce qui leur
permet de n'étudier qu'une variable binaire.

\section{L'analyse automatique ("minage") de projets de logiciel libre}

Pour leur analyse quantitative, \textcite{signals-2019} commencent par collecter des données sur la plateforme
d'hébergement de code source \gls{github} via son \gls{API}. Ce choix est cohérent avec un grand nombre
d'études de ces dernières années \sideparencite{github-mapping-2017}, la plateforme est très populaire dans le
milieu du logiciel libre et a connu une croissance très importante ces dernières années. Ces mêmes études
notent cependant parfois que se restreindre à cette seule plateforme est une limite diminuant la
représentativité de l'échantillon qu'elles étudient \sideparencite{swh-growth-2019}. Les limites et erreurs
courantes liées au "minage" de \gls{github} font d'ailleurs l'objet d'une littérature grandissante et
remettant potentiellement en question les résultats des études utilisant cette technique sans détailler
suffisamment leur méthodologie de collecte des données \sideparencite{mining-github-2014,penumbra-oss-2022}.

Le nombre de projets et d'artéfacts qui leur sont lié est par ailleurs en augmentation rapide, voir
exponentielle. \sidetextcite{swh-growth-2019} estiment que le nombre de \glspl{commit} uniques originaux
double environ tous les trente mois, le nombre de fichiers uniques originaux doublerait quant à lui tous les
vingt-deux mois. Cette explosion du nombre de \glspl{commit} et de fichiers s'accompagne d'un important
phénomène de duplication : de nombreux fichiers se retrouvent dans de nombreux projets différents, comme les
notices de licence par exemple. Les \glspl{commit} eux-même peuvent se trouver dans plusieurs projets
différents, c'est ce qui arrive lorsqu'un projet débute en tant que \gls{fork} d'un autre. Ces deux points
rendent de plus en plus difficile l'analyse efficace et à grande échelle de l'historique de développement des
projets de logiciel libre.

\subsection{L'archive de logiciels de Software Heritage}
\label{ssec:swh-graph}

Les techniques de compression de graphe sont aujourd'hui suffisamment avancées pour permettre une nouvelle
approche à cette analyse automatique des projets, en rendant un très grand nombre de ceux-ci disponibles au
sein d'une même structure de donnée unifiée et optimisée \sideparencite{swh-graph-2020}. Cette approche permet
par exemple de grandement faciliter la détection des "\glspl{clone}" ou "\glspl{fork}" de projets, c'est à
dire des projets ayant une part de leur historique de développement en commun. La structure de donnée proposée
par \textcite{swh-graph-2020} est un graphe orienté où chaque \gls{commit} est représenté par un nœud et où
chaque arc $u \xrightarrow{succ} v$ ("$v$ est un successeur de $u$", autrement dit pour les \glspl{commit}
"$v$ est un ancêtre immédiat de $u$") possède aussi un arc inverse $u \xleftarrow{pred} v$ ("$u$ est un
prédécesseur de $v$"). Une telle structure permet donc de trouver tous les "\gls{fork}" d'un projet avec un
classique calcul de composante fortement connexe, ce qui peut se faire avec un simple parcours profondeur ou
largeur du graphe à partir de n'importe lequel de ses \glspl{commit}.

Software Heritage\sidenote{\url{https://www.softwareheritage.org/}} est une initiative exploitant ces
techniques de compression afin de construire une archive aussi complète que possible du code source
actuellement disponible publiquement dans le monde. L'archive est elle aussi publique et un de leurs objectifs
est de la rendre exploitable pour la recherche \sideparencite{swh-2017}. Il s'agit du corpus que
\textcite{swh-graph-2020} ont utilisé comme exemple pour leur modèle de compression. L'archive comptait en
2018 plus d'un milliard de \glspl{commit} uniques archivés depuis quatre-vingt-cinq millions d'origines
différentes (telles que des \en{remotes} \gls{git} ou des versions de paquets Python publiés sur
PyPi\sidenote{\url{https://pypi.org/}}). L'archive est stockée dans une structure en "ajout seul" : elle
grandit de façon continue en archivant petit à petit de nouveaux projets et en ajoutant de nouveaux artéfacts
à chaque évolution des projets déjà archivés, ce qui en fait l'un des corpus les plus complets et exploitable
par la recherche scientifique \sideparencite{swh-2019,swh-growth-2019}.

\section{Méthodologies de recherches autour de ces questions}

\subsection{Méthodes qualitatives d'identification des barrières d'entrée}

\textcite[p.~1006]{barriers-2018} analysent les réponses écrites lors d'entretiens via un codage qualitatif
suivant un modèle de \sidetextcite{barriers-2014}, avec pour objectif de répondre à trois questions de
recherche que l'on pourrait traduire ainsi : "Quels problèmes peuvent être révélés en regardant les logiciels
libres au travers des outils et de l'infrastructure ?", "Les outils et l'infrastructure participent-ils à
créer des barrières d'entrée pour les nouvelles personnes souhaitant contribuer ? Si oui, comment ?" et
"Existe-t-il des barrières d'entrées pour ces personnes qui soient biaisés sur la question du genre ? Si oui,
de quelles façons sont-elles biaisées ?". L'entretien lui-même se fait en suivant un processus appelé
"\en{GenderMag}", lui-même un type de "\en{Cognitive Walkthrough}" qui consiste à présenter aux sujets
interrogés un personnage fictif avec ses traits de caractères, ses compétences et ses préférences (une
"\emph{persona}") et à leur demander d'imaginer comment ce personnage se comporterait dans une situation
donnée. \sideparencite{gendermag-2016,cognitive-walkthrough-2000}

\subsection{Sélection des projets étudiés}

Dans une méta-analyse concernant l'étude des barrières d'entrée, \sidetextcite{barriers-meta-2015} remarquent
un biais courant dans la sélection des projets de logiciel libre retenus dans la littérature (p. 83) : les
auteurs ont tendance à plutôt étudier les projets importants, matures et populaires, car ils sont plus
susceptibles d'apporter un volume important de données. Bien que les résultats semblent cohérent avec les plus
rares études sélectionnant de petits projets, ils invitent la communauté scientifique à s'y intéresser plus en
profondeur et à constituer des échantillons plus variés.

Quand \sidetextcite{mining-github-2014} analysent les \glspl{pull request} sur \gls{github}, ils ne retiennent
que les projets ayant au moins deux auteurs de \glspl{commit} uniques, ce qu'ils considèrent comme des projets
"vraiment collaboratifs". Ce choix vient de leur observation qu'une très grande partie des projets hébergés
sur la plateforme ($71\%$) sont des projets "personnels", utilisés par leurs auteurs à des fins
d'expérimentation ou de stockage, sans réelle ouverture à la collaboration.
