\chapter*{Conclusion}

Les problématiques liées au numérique prennent de plus en plus d'importance dans notre monde, la question du
logiciel libre devient de ce fait un enjeu de société déterminant pour la transparence des technologies,
l'ouverture de l'information et la collaboration international. Rendre disponible publiquement et gratuitement
le code source et son processus de développement ne suffit cependant pas à le rendre réellement accessible, de
nombreuses barrières existent pour les personnes souhaitant se familiariser avec les projets de logiciel libre
et y contribuer. Ces barrières représentent même un frein pour les enseignants souhaitant introduire leurs
étudiants à ce milieu, même lorsque ces enseignants lui sont eux-mêmes déjà familiers.

Dans ce mémoire, nous avons commencé la construction d'un outil permettant aux enseignants de répondre à l'un
de ces freins : la sélection de projets de logiciel libre réels susceptibles d'être un bon support de travaux
pratiques. Ce travail préliminaire se base sur une littérature florissante d'identification des barrières aux
contribution dans le logiciel libre, sa méthodologie s'inscrit quant à elle dans le domaine de l'analyse
automatique des dépôts logiciels (\href{https://conf.researchr.org/series/msr}{\en{Mining Software
Repositories}}).

Par l'analyse de l'archive de Software Heritage, l'un des corpus de développement logiciel les plus
représentatifs et exploitables dans un contexte de recherche, nous avons tenté de déterminer si trois signaux
facilement observables pour un enseignant ou un nouveau contributeur sont prédictifs de la capacité d'un
projet à suffisamment accompagner les nouveaux contributeurs pour leur permettre de mener leur première
contribution à terme.

Nos résultats suggèrent que la présence d'instructions de contribution au sein d'un projet (typiquement au
travers d'un fichier "\texttt{CONTRIBUTING.md}"), ainsi que le nombre de contributeurs uniques ayant récemment
contribué au projet, sont positivement corrélés au nombre de \emph{nouveaux} contributeurs étant parvenus à
ajouter leur pierre à l'édifice. Nous n'avons en revanche pas trouvé de corrélation, positive ou négative,
permettant d'affirmer que le nombre de \glspl{commit} récents au sein d'un projet était prédictif du nombre de
nouveaux contributeurs allant au bout de leur contribution.

Quelques précautions mériteraient cependant d'être prises dans l'interprétation de ces résultats. La limite la
plus importante de ce travail est l'absence d'information permettant d'inférer un quelconque lien de causalité
entre les éléments étudiés. Rien dans nos résultats ne nous permet par exemple de dire que la présence
d'instructions de contribution \emph{améliore} l'accessibilité d'un projet de logiciel libre pour les nouveaux
contributeurs, la corrélation que nous observons entre ces deux variables pourrait s'expliquer par le fait que
les mainteneurs d'un projet qui ont tendance à efficacement guider les nouveaux contributeurs ont aussi
tendance à écrire des instructions de contribution. Par ailleurs, plusieurs propriétés statistiques des
données collectées rendent difficile leur analyse. Nous avons tenté traiter ces difficultés au mieux de nos
capacités, mais une étude et une préparation plus rigoureuse des données pourraient, au mieux, améliorer la
fiabilité des conclusions que nous en avons tiré, ou au pire, les invalider.

Les perspectives de recherche concernant l'accessibilité des projets de logiciel libre sont encore nombreuses,
le sujet étant à la fois très changeant et relativement nouveau, la littérature existe mais est encore très
qualitative. De futures études quantitatives pourraient s'intéresser aux éventuels mécanismes causaux
responsables de l'accessibilité observée des projets. Le domaine de l'analyse automatique des dépôts logiciels
pose encore beaucoup de questions intéressantes, l'archive de Software Heritage par exemple apporte une
nouvelle méthode d'observation particulièrement puissante, mais celle-ci est encore jeune et peu d'étude se
consacrent aux bénéfices et limites de ce qu'elle peut apporter à la science. Enfin, si nos résultats
suggèrent que la présence d'instructions de contribution ainsi que le nombre de contributeurs uniques récents
sont de bons indicateurs à observer pour un enseignant souhaitant trouver un projet sur lequel faire
travailler ses étudiants, de futures recherches mériteraient d'être conduites afin de déterminer si cette
accessibilité se traduit effectivement dans la qualité de l'apprentissage, ou si le contexte est trop
différent pour que les étudiants soient réduit à la catégorie générique des "nouveaux contributeurs".
