% Book information
\titlehead{}
\title{L'identification des projets de logiciel libre accessibles aux nouveaux contributeurs}
\subtitle{Mémoire de Master Sciences de l'Éducation, parcours SYNVA}
\author[PH]{Soutenu par\\Paul HERVOT}
\date{le 8 septembre 2022}
\publishers{
    Mémoire présenté en vue de l'obtention du Grade de Master

    \bigskip
    \bigskip
    \bigskip
    \bigskip

    Commission de soutenance composée par :

    \bigskip

    \begin{tabular}{lr}
        Benoît CRESPIN & directeur\\
        Richard NGU LEUBOU & assesseur
    \end{tabular}

    \bigskip

    \begin{figure}[ht]
        \includegraphics[width=0.45\textwidth]{inspe_unistra}
        \includegraphics[width=0.45\textwidth]{unistra}
    \end{figure}
}
\dedication{%
    Un grand merci à Antoine Pietri pour son aide précieuse lors de la collecte des données sur le graphe de
    Software Heritage.

    \emph{Deux} grands mercis à Marc Trestini et Benoît Crespin, ayant par la force des choses tous deux
    endossé le rôle de directeur de ce mémoire à différentes étapes de son écriture, leurs expertises
    complémentaires m'a permis d'arriver au bout de ce travail si nouveau pour moi.%
}

% Pre-document content
\setchapterstyle{plain}
\pagelayout{wide}

% Copyright page
\makeatletter
\uppertitleback{\@titlehead}
\lowertitleback{
    \textbf{Pas de copyright} \\
    \ccby\ Ce mémoire est mis à disposition selon les termes de la
    \href{https://creativecommons.org/licenses/by/3.0/fr/}{Licence \en{Creative Commons Attribution}}.

    \medskip

    \textbf{Colophon} \\
    Ce document a été composé avec l'aide de
    \href{https://sourceforge.net/projects/koma-script/}{\KOMAScript} et
    \href{https://www.latex-project.org/}{\LaTeX} en utilisant la classe
    \href{https://github.com/fmarotta/kaobook/}{kaobook}.

    \medskip

    \textbf{Éditeur} \\
    Première publication le \todo[inline]{TODO} par l'Université de Strasbourg
}
\makeatother

% TODO: should it be included in the PDF?
% \includepdf[pages=-]{contrat_diffusion.pdf}

% Note that \maketitle outputs the pages before here
\maketitle

% Table of contents, figures and tables
\begingroup % Local scope for the following commands
    % Replace the automatic page break between those lists by a simple vertical split
    \newcommand{\biggerskip}{\vspace{4\bigskipamount}}
    \let\cleardoublepage\biggerskip
    \let\clearpage\biggerskip

    \tableofcontents
    \listoffigures
    \listoftables
\endgroup


% Main document content
\pagestyle{scrheadings}
\setchapterstyle{kao}

\pagelayout{margin}
\chapter{Introduction}

\section{Parcours personnel}

TODO, notes :

\begin{itemize}
    \item apprend les bases du code en terminale
    \item école d'ingénieur spécialisée en informatique (\gls{epita})
    \item associations Prologin et GConfs -> développement en équipe avec beaucoup de turnover (asso
        étudiante) + fort intérêt pour le partage de connaissance
    \item le \gls{fosdem} comme rendez-vous annuel
    \item rôle d'assistant pédagogique -> fort goût pour l'éducation
    \item service civique en collège/lycée pour confirmer ce goût
    \item enseignant à \gls{epita} (mentionner la première année d'administration ?)
    \item curiosité naissante pour la recherche (discussions avec des amis en thèse ou passionnés de
        méthodologie scientifique)
    \item formation \gls{synva} pour bases théoriques en enseignement, aidé de mes forces en informatique, et pour
        mettre un pied dans la porte de la recherche
    \item quelques créations et contributions à des projets de logiciel libre depuis le début de mes études
        supérieures (projets Prologin et GConfs, plugins vim, tentative citra, alacritty, nixpkgs, plugin
        thunderbird, paquets \LaTeX, \ldots)
\end{itemize}

\section{Le logiciel libre}

\todo[inline]{TODO : quelques définitions, notamment \href{https://opensource.org/osd}{open source / logiciel libre}.}

\section{Les projets de logiciel libre dans l'enseignement de l'informatique}

La pédagogie de projet est une méthode d'enseignement qui consiste à inviter les apprenants à appliquer leurs
connaissances et à en acquérir de nouvelles au travers d'un projet plus ou moins concret et imitant plus ou
moins une situation réelle. Réalisé soit individuellement soit en groupe, les projets aboutissent généralement
à une production évaluée par l'enseignant, parfois au travers d'une présentation donnée par les apprenants.
L'efficacité de cette méthode a fait l'objet de nombreuses recherches au cours des années précédentes.
Plusieurs méta-analyses concluent que cette méthode a des effets mesurables, positifs et importants sur les
résultats académiques des apprenants en sciences sociales et en sciences naturelles
\sideparencite[-3cm]{pbl-2019, pbl-2018}.

Les projets proposés aux étudiants sont le plus souvent des projets "jouets" imaginés par les enseignants dans
le seul but de servir d'exercice. Utiliser au contraire des projets réels, qui ne cessent pas d'exister après
la fin de la séquence pédagogique, sur lesquels faire travailler les étudiants donne des résultats encore
meilleurs, mais est aussi plus difficile et incertain à encadrer pour les enseignants
\sideparencite{real-pbl-2010, real-pbl-2004}. Il est notamment très difficile pour l'enseignant de prévoir à
l'avance les problèmes que les apprenants risquent de rencontrer, c'est pourquoi d'autres efforts dans cette
direction ont tenté un compromis, consistant à créer des projets de logiciel libre aussi proches des
situations réelles que possible, mais dédiés à l'éducation \sideparencite{oss-edu-2008}. Les projets en
question ne sont cependant plus accessibles aujourd'hui, une explication possible à leur disparition étant
leur portée réduite à d'éducation.

Bien qu'elles soient plutôt rares, quelques initiatives existent aussi pour enseigner spécifiquement les
pratiques du logiciel libre dans l'éducation supérieur et la formation tout au long de la vie. Certaines sont
des initiatives venant des communautés du logiciel libre, comme le \en{Professional Certificate in Open Source
Software Development, Linux and Git}%
\sidenote{\url{https://www.edx.org/professional-certificate/linuxfoundationx-open-source-software-development-linux-and-git}}
de la \en{Linux Foundation}, et en particulier la première séquence : \en{Open Source Software Development:
Linux for Developers}%
\sidenote{\url{https://www.edx.org/course/open-source-software-development-linux-for-developers}} ; ou les
séminaires de l'\en{open source initiative}\sidenote{\url{https://opensource.org/osi-open-source-education}}.
Enfin, certaines universités proposent des cursus informatique ayant des éléments visant spécifiquement les
pratiques du logiciel libre, c'est le cas notamment de l'Université de l'État de Caroline du Nord aux
États-Unis, qui a expérimenté plusieurs façons d'inclure la contribution aux projets de logiciel libre dans
leurs cursus d'informatique \sideparencite{oss-edu-2008, oss-edu-2007} ; mais aussi de l'université de Calais
qui propose actuellement un "Master Informatique - Ingénierie du logiciel libre"%
\sidenote{\url{https://www.univ-littoral.fr/formation/offre-de-formation/masters/master-informatique-ingenierie-du-logiciel-libre/}},
sous la forme d'une formation en alternance.

\pagelayout{wide}

\addpart{Partie théorique}

\pagelayout{margin}
\chapter{Revue de littérature}

\sidetextcite{strategies-2012} identifient 5 grandes étapes dans les intéraction que les entreprises ont
généralement avec les logiciels libres :
\begin{description}
    \item[identification] : évaluer la qualité d'un logiciel libre, ses garanties et les potentiels brevets
        utilisés ;
    \item[adoption] : utilisation du logiciel ;
    \item[conformité] : examen des licence attachées au logiciel, de leur compatibilité avec les activités de
        l'entreprise, et des contraintes qu'elles apportent dans les activités futures ;
    \item[contribution] : partage avec la communauté.
\end{description}

\textcite{strategies-2012} notent un grand besoin de formation au sein des entreprises sur le sujet des
licences, ainsi qu'un intérêt pour celle-ci à participer aux projets qu'elles utilisent.

Au sein d'un projet de logiciel, la communité produit généralement plus de \englpl{commit} que de messages sur
les listes de diffusion, mais il y a une corrélation entre le nombre de \englpl{commit} et le nombre de
message que chaque développeur fait \sideparencite{contribution-patterns-2010}. Cela indique qu'au sein d'un
projet de logiciel libre, bien que la production de \englpl{commit} prend plus de place, les contributeurs
participent aux échanges sur les listes de diffusion proportionnellement au nombre de \englpl{commit} qu'ils
font. Si l'on souhaite enseigner ces comportements aux étudiants, il ne faut donc pas négliger cet aspect
communication.

\sidetextcite{barriers-2018} notent la difficulté des nouveaux volontaires à rejoindre une communauté de
logiciel libre, citant comme exemple extrême le projet \en{Apache Hadoop} voyant 82\% de ses nouveau
volontaire quitter le projet après leur première contribution \sideparencite{hadoop-dropout-2013} (TODO: lire
cet article en détail et comprendre les nuances).

\pagelayout{wide}

\addpart{Partie expérimentale}

\chapter{Méthodologie}

\newtheorem{hypo}{Hypothèse}
\newcommand{\newhyp}[2]{%
    \begin{restatable}{hypothesis}{#1}
        \label{hyp:#1}#2
    \end{restatable}%
}

\todo[inline, color=yellow]{%
    Mon intuition est de trouver une ou plusieurs métriques dans les projets de logiciel libre pouvant être
    prédictives de leur tendance à bien accueillir les nouveaux contributeurs (que l'on pourrait par exemple
    mesurer en regardant la régularité avec laquelle de nouveaux noms apparaissent dans le git log). De tels
    facteurs pourraient servir lors de la préparation de cours consistant à participer à un projet de logiciel
    libre, l'enseignant·e pourrait identifier les projets les plus susceptible de convenablement accueillir
    ses étudiant·es.%
}

\section{Hypothèses}

\newhyp{contributing.md}{%
    les projets possédant des instructions de contribution (fichier type "\en{CONTRIBUTING.md}" ou section
    type "\en{Contributing}" dans le \en{README}) sont plus accessibles pour les nouveaux contributeurs que
    ceux n'en ayant pas \sideparencite[][voir][p.~11]{signals-2019}.%
}

\newhyp{contributorcount}{%
    le nombre de contributeurs récents d'un projet influe sur son accessibilité pour les nouveaux
    contributeurs \parencite[voir][p.~12-13]{signals-2019}.%
}

\newhyp{recentcommits}{%
    le nombre de \englpl{commit} \emph{récents} au sein d'un projet influe sur son accessibilité pour les
    nouveaux contributeurs \parencite[voir][p.~13]{signals-2019}.
}

\section{Mesure de l'accessibilité choisie}

\todo[inline]{%
    TODO : méthode choisie pour mesurer à quel point un projet est accessible pour les nouveaux
    contributeurs.%
}

\todo[inline, color=yellow]{%
    Idée : regarder chaque mois (ou trimestre ? ou année ?) le pourcentage de contributeurs (auteurs uniques
    de \engl{commit}) qui contribue dans ce projet pour la première fois. Malheureusement pour l'instant je
    n'ai vu personne essayer de mesurer ce genre de chose donc il n'y a que mon intuition pour me dire que
    c'est une mesure correcte de l'accessibilité pour les nouveaux contributeurs.%
}

\section{Constitution de l'échantillon}

L'échantillon de départ est constitué de la totalité des projets archivés dans
le graph de Software Heritage. Plusieurs critères d'exclusion ont ensuite été
appliqués :

\begin{itemize}
    \item quand deux projets ou plus ont des \englpl{commit} en commun, seul
        celui qui a reçu le plus d'activité (mesuré en regardant le nombre de
        \englpl{commit} entre le premier et le dernier) a été retenu ;
    \item les projets n'ayant enregistré aucune activité (aucun commit) entre le
        1er janvier 2021 et le 1er juin 2021 n'ont pas été retenus ;
\end{itemize}

\section{Collecte des données}

\todo[inline]{%
    TODO : parcours du graph de Software Heritage avec l'API graph
    (\url{https://docs.softwareheritage.org/devel/swh-graph/index.html}) soit sur un sous-graph compressé
    (\url{https://docs.softwareheritage.org/devel/swh-dataset/graph/dataset.html}) qui tient sur mon
    ordinateur personnel, soit sur le graph complet si je trouve quelqu'un pour me donner un peu de temps de
    calcul sur un gros serveur.%
}

\todo[inline]{TODO : penser à mettre le code en annexe}

\section{Traitement des données}

\todo[inline]{TODO}

\todo[inline]{%
    TODO bonus : faire un "\en{replication package}" avec mon code, comme le font \textcite{swh-graph-2020}
    ici : \url{https://zenodo.org/record/3574459} ?%
}


\pagelayout{wide}
\setchapterstyle{kao}
\chapter{Résultats}

% Figure setup for
\captionsetup[figure]{format=plain,singlelinecheck=true,justification=centering}
\captionsetup[subfigure]{format=plain,singlelinecheck=true,justification=centering}

\todo[inline]{TODO : rédiger}

L'analyse du graphe de Software Heritage a permis de déterminer les quatre variables de recherche pour $TODO$
projets distincts dont les historique de développement sont eux aussi entièrement distincts deux à deux (aucun
projet analysé n'est un \en{fork} d'un autre).

Un aperçu initial des données collectées (figure~\ref{fig:data_description}) révèle quelques propriétés
intéressantes de la population étudiée. Les projets possédant des instructions de contribution ($TODO$, soit
$TODO\%$ des projets) sont significativement moins nombreux que ceux n'en possédant pas. L'écart entre la
moyenne et la médiane du nombre de \englpl{commit} récents, ainsi que sont écart type, indiquent une forte
variation de ce nombre au sein de la population étudiée, mais aussi la présence de quelques individus
extrêmes, avec un maximum à TODO \englpl{commit} récents observés dans un seul projet. Cette dernière
observation se retrouve aussi dans le nombre de \emph{contributeurs} récents, bien que de façon moins
spectaculaire. Rappelons à la vue des quantiles de cette dernière valeur que les projets ayant vu moins de
deux contributeurs distincts récents ont été exclus de l'étude (voir section
\ref{sec:constitution_echantillon} p.~\pageref{sec:constitution_echantillon}).

\begin{figure}
    \centering
    \begin{tabular}{cc}
        \begin{tabular}{ll}
 & \textbf{hasContrib} \\
count & 60966 \\
unique & 2 \\
top & no \\
freq & 52394 \\
\end{tabular}
 &
        \begin{tabular}{lr}
 & \textbf{recentCommitCount} \\
count & 60966.00 \\
mean & 61.79 \\
std & 408.96 \\
min & 2.00 \\
25\% & 8.00 \\
50\% & 19.00 \\
75\% & 49.00 \\
max & 36176.00 \\
\end{tabular}

        \\
        \begin{tabular}{lr}
 & recentContributorCount \\
count & 27619.000000 \\
mean & 3.217640 \\
std & 6.641740 \\
min & 2.000000 \\
25\% & 2.000000 \\
50\% & 2.000000 \\
75\% & 3.000000 \\
max & 522.000000 \\
\end{tabular}
 &
        \begin{tabular}{lr}
 & \textbf{newContributorCount} \\
count & 60966 \\
mean & 0.52 \\
std & 1.66 \\
min & 0 \\
25\% & 0 \\
50\% & 0 \\
75\% & 1 \\
max & 130 \\
\end{tabular}

    \end{tabular}

    \caption{Aperçu statistique des données collectées}
    \label{fig:data_description}
\end{figure}

\begin{figure}
    \begin{subfigure}[t]{0.5\textwidth}
        \includegraphics[width=\textwidth]{experiment/data_analysis/hasContrib_Count}
        \caption{Nombre de projets avec ou sans\\instructions de contribution}
    \end{subfigure}%
    \begin{subfigure}[t]{0.5\textwidth}
        \includegraphics[width=\textwidth]{experiment/data_analysis/recentCommitCount_distribution}
        \caption{Nombre de \englpl{commit} récents\\(ordonnées logarithmiques)}
    \end{subfigure}

    \begin{subfigure}[t]{0.5\textwidth}
        \includegraphics[width=\textwidth]{experiment/data_analysis/recentContributorCount_distribution}
        \caption{Nombre de contributeurs récents\\(ordonnées logarithmiques)}
    \end{subfigure}%
    \begin{subfigure}[t]{0.5\textwidth}
        \includegraphics[width=\textwidth]{experiment/data_analysis/newContributorCount_distribution}
        \caption{Nombre de nouveaux contributeurs\\(ordonnées logarithmiques)}
    \end{subfigure}

    \caption{Distribution des individus selon chaque variable numérique}
    \label{fig:distribution}
\end{figure}

\begin{figure}
    \begin{subfigure}[t]{0.8\textwidth}
        \includegraphics[width=\textwidth]{experiment/data_analysis/hasContrib_meanNewContributorCount}
        \caption{Moyenne du nombre de nouveaux contributeurs pour chaque catégorie}
    \end{subfigure}

    Test de Wilcoxon-Mann-Whitney : $U = 2.7308169 \times 10^{8}$ ($p = 1.1549639 \times 10^{-140}$, $ρ = 0.56673768$)
    \caption{Effet de la présence d'instructions de contribution}
    \label{fig:hasContrib}
\end{figure}

Ces deux catégories de projets ont été comparées avec le test de Mann-Whitney. Celui-ci donne une valeur $U =
TODO$ avec une taille d'effet $ρ = TODO$, ce qui signifie en langage courant que si l'on choisi au hasard un
projet $A$ possédant des instructions de contribution et un projet $B$ n'en possédant pas, il y a environ
$TODO\%$ de chances que le projet $A$ ait vu un plus grand nombre de nouveaux contributeurs durant la période
étudiée que le projet $B$. Le test confirme en outre avec un degré de confiance supérieur à $TODO\%$ ($p <
TODO$) que la distribution des valeurs au sein de ces deux catégories (projets avec instructions de
contribution ou sans) est bien différente, avec un avantage pour les projets avec instructions de
contribution.

\begin{figure}[ht]
    \centering
    \begin{subfigure}[t]{0.8\textwidth}
        \includegraphics[width=\textwidth]{experiment/data_analysis/recentContributorCountRegression_linearScale}
        \caption{Échelle linéaire}
    \end{subfigure}

    \begin{subfigure}[t]{0.8\textwidth}
        \includegraphics[width=\textwidth]{experiment/data_analysis/recentContributorCountRegression_logScale}
        \caption{Échelle logarithmique}
    \end{subfigure}

    $\mathit{newContributorCount} = \mathit{recentContributorCount} \times 0.19790996 - 0.1295073$\\($R^2 = 0.50784721$)
    \caption{Nombre de nouveaux contributeurs en fonction du nombre de contributeurs récents uniques}
    \label{fig:contributorCount}
\end{figure}

\begin{figure}[ht]
    \centering
    \begin{subfigure}[t]{0.5\textwidth}
        \includegraphics[width=\textwidth]{experiment/data_analysis/recentCommitCountRegression_linearScale}
        \caption{Échelle linéaire}
    \end{subfigure}%
    \begin{subfigure}[t]{0.5\textwidth}
        \includegraphics[width=\textwidth]{experiment/data_analysis/recentCommitCountRegression_logScale}
        \caption{Échelle logarithmique}
    \end{subfigure}

    $\mathit{newContributorCount} = \mathit{recentCommitCount} \times 0.0013121815 + 0.44112747$\\($R^2 = 0.10410059$)
    \caption{Nombre de nouveaux contributeurs en fonction du nombre de \englpl{commit} récents}
    \label{fig:commitCount}
\end{figure}

\pagelayout{margin}


\addpart{Conclusion}

\chapter*{Conclusion}

Les problématiques liées au numérique prennent de plus en plus d'importance dans notre monde, la question du
logiciel libre devient de ce fait un enjeu de société déterminant pour la transparence des technologies,
l'ouverture de l'information et la collaboration international. Rendre disponible publiquement et gratuitement
le code source et le processus de développement ne suffit cependant pas à le rendre réellement accessible, de
nombreuses barrières existent pour les personnes souhaitant se familiariser avec les projets de logiciel libre
et y contribuer. Ces barrières représentent même un frein pour les enseignants souhaitant introduire leurs
étudiants à ce milieu, même lorsque ces enseignants y sont eux-mêmes déjà familiers.

Dans ce mémoire, nous avons commencé la construction d'un outil permettant aux enseignants de répondre à l'un
de ces freins : la sélection de projets de logiciel libre réels susceptibles d'être un bon support de travaux
pratiques. Ce travail préliminaire se base sur une littérature florissante d'identification des barrières aux
contribution dans le logiciel libre, sa méthodologie s'inscrit quant à elle dans le domaine de l'analyse
automatique des dépôts logiciels (\href{https://conf.researchr.org/series/msr}{\en{Mining Software
Repositories}}).

Par l'analyse de l'archive de Software Heritage, l'un des corpus de développement logiciel les plus
représentatifs et exploitables dans un contexte de recherche, nous avons tenté de déterminer si trois signaux
facilement observables pour un enseignant ou un nouveau contributeur sont prédictifs de la capacité d'un
projet à accompagner les nouveaux contributeurs suffisamment pour leur permettre de mener leur première
contribution à terme.

Nos résultats suggèrent que la présence d'instructions de contribution au sein d'un projet (typiquement au
travers d'un fichier "\texttt{CONTRIBUTING.md}"), ainsi que le nombre de contributeurs uniques ayant récemment
contribué au projet, est positivement corrélé au nombre de \emph{nouveaux} contributeurs étant parvenus à
ajouter leur pierre à l'édifice. Nous n'avons en revanche pas trouvé de corrélation, positive ou négative,
permettant d'affirmer que le nombre de \glspl{commit} récents au sein d'un projet était prédictif du nombre de
nouveaux contributeurs allant au bout de leur contribution.

Quelques précautions mériteraient cependant d'être prises dans l'interprétation de ces résultats. La limite la
plus importante de ce travail est l'absence d'information permettant d'inférer un quelconque lien de causalité
entre les éléments étudiés. Rien dans nos résultats ne nous permet par exemple de dire que la présence
d'instructions de contribution \emph{améliore} l'accessibilité d'un projet de logiciel libre pour les nouveaux
contributeurs, la corrélation que nous observons entre ces deux variables pourrait s'expliquer par le fait que
les mainteneurs d'un projet qui ont tendance à efficacement guider les nouveaux contributeurs ont aussi
tendance à écrire des instructions de contribution. Par ailleurs, plusieurs propriétés statistiques des
données collectées rendent difficile leur analyse. Nous avons tenté traiter ces difficultés au mieux de nos
capacités, mais une étude et une préparation plus rigoureuse des données améliorerait sans doute la fiabilité
des conclusions que nous en avons tiré.

Les perspectives de recherche concernant l'accessibilité des projets de logiciel libre sont encore nombreuses,
le sujet étant à la fois très changeant et relativement nouveau, la littérature existe mais est encore très
qualitative. De futures études quantitatives pourraient s'intéresser aux éventuels mécanismes causaux
responsables de l'accessibilité observée des projets. Le domaine de l'analyse automatique des dépôts logiciels
pose encore beaucoup de questions intéressantes, l'archive de Software Heritage par exemple apporte une
nouvelle méthode particulièrement puissante, mais celle-ci est encore jeune et peu d'étude se consacrent aux
bénéfices et limites de ce qu'elle peut apporter à la science. Enfin, si nos résultats suggèrent que la
présence d'instructions de contribution ainsi que le nombre de contributeurs unique récents sont de bons
indicateurs à observer pour un enseignant souhaitant trouver un projet sur lequel faire travailler ses
étudiants, de futures recherches mériteraient d'être conduites afin de déterminer si cette accessibilité se
traduit effectivement dans la qualité de l'apprentissage, ou si le contexte est trop différent pour que les
étudiants soient réduit à la catégorie générique des "nouveaux contributeurs".


% Number chapters with letters from now on
\appendix

\addpart{Annexes}

\chapter{Code d'extraction des données}

\chapter{Code de calcule des résultats}


% End of the main document content
\backmatter

\setchapterstyle{plain}

\tableofcontents

\defbibnote{bibnote}{Références utilisées, présentées par ordre de première citation.\par\bigskip}
\printbibliography[heading=bibintoc, prenote=bibnote]

\printindex

\printglossary


