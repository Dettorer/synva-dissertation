\setchapterstyle{plain}

\defbibnote{bibnote}{Références utilisées, présentées par ordre de première citation.}

\printbibliography[heading=bibintoc, prenote=bibnote]

\printindex

\printglossary

\pagebreak

\section*{Résumé}

Ce mémoire entreprend d'identifier quels indicateurs facilement observables par un acteur individuel peuvent
servir à estimer l'accessibilité d'un projet de logiciel libre pour les personnes souhaitant y contribuer pour
la première fois. L'ubiquité des technologies numériques dans nos vies pose de plus en plus la question de la
place du logiciel libre et de son intérêt pour la transparence des logiciels et leur gestion démocratique,
pour autant le simple fait de rendre le code source d'un logiciel publique ne le rend pas automatiquement
accessible, et les nouveaux contributeurs de ces projets rencontrent de nombreuses barrières d'entrée les
empêchant parfois de mener leurs contributions à leur terme. Au travers d'une analyse à grande échelle de
l'archive de Software Heritage, nous testons la pertinence de trois indicateurs dans l'identification des
projets de logiciel libre accessibles aux nouveaux contributeurs. Nos résultats montrent un lien de
corrélation positif entre le nombre de contributions menées à leur terme au sein d'un projet par de nouveaux
contributeurs et la présence d'instructions de contribution, ainsi qu'une corrélation positive entre ce même
nombre et le nombre de contributeurs uniques récents du projet, ils ne permettent en revanche pas de mettre en
évidence un quelconque lien avec le nombre de \glspl{commit} récents. De tels indicateurs trouveraient leur
utilité notamment dans l'enseignement des pratiques du logiciel libre, les enseignants de ce sujet ayant
souvent du mal à sélectionner des projets accessibles sur lesquels faire travailler les étudiants.

\textbf{%
    Mots-clés : Logiciel libre, Analyse automatique de dépôts logiciels, Barrières d'entrée du logiciel libre,
    Nouveaux contributeurs.
}

\begin{otherlanguage}{english}
    \section*{Abstract}

    This dissertation tries to identify which signals easily observable by an individual can help estimate the
    accessibility of a FOSS project for people who want to contribute to it for the first time. The ubiquity
    of digital technology in our lives asks the question of the role of Free and Open Source Software in it
    and its benefits for software transparency and their democratic governance. However, simply making the
    source code of a software public doesn't automatically make it accessible, and new contributors
    approaching these projects often face many barriers to entry that prevent them from bringing their
    contribution to completion. Through a large-scale software mining of the Software Heritage archive, we
    test the pertinence of three signals in the identification of accessible FOSS projects for new
    contributors. Our results show a positive correlation between the number of new contributors of a project
    successfully bringing their contribution to completion and the presence of contributing guidelines, as
    well as a positive correlation between that same number and the number of recent unique contributors in
    the project; on the other hand, they show no evidence for any link regarding the number of recent commits.
    Such signals could find a use in the teaching of FOSS practices, as teachers of this subject often find it
    difficult to select an accessible project for their students to work on.

    \textbf{Keywords: FOSS, Mining Software Repositories, Open source barriers to entry, New contributors.}
\end{otherlanguage}
