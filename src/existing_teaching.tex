\chapter{Démarches existantes}

\begin{marginfigure}
    \href{https://www.linuxfoundation.org/}{\includesvg[width=\marginparwidth]{linux-foundation-vert-color}}
    \caption{Logo de la \english{Linux Foundation}}
    \labfig{LFlogo}
\end{marginfigure}

De la \english{Linux Foundation} :
  \href{https://www.edx.org/professional-certificate/linuxfoundationx-open-source-software-development-linux-and-git}
  {\english{Professional Certificate in Open Source Software Development, Linux and Git}}
et en particulier la première séquence :
  \href{https://www.edx.org/course/open-source-software-development-linux-for-developers}
  {\english{Open Source Software Development: Linux for Developers}}.

\begin{description}
    \item[\english{the who}] : exemples de projets et communautés connues
    \item[\english{the what}] : définitions
    \item[\english{the why}] : les bénéfices
    \item[\english{the how}] : les licenses, la conformité, conseils de collaboration, gestion de la
        diversité, développement continu et intégration
\end{description}

Les deux autres séquences s'intéressent d'un point de vue général au développement Linux et et à git.
