\chapter{Formations existantes}

\begin{marginfigure}
    \href{https://www.linuxfoundation.org/}{\includesvg[width=\marginparwidth]{linux-foundation-vert-color}}
    \caption{Logo de la \en{Linux Foundation}}
    \labfig{LFlogo}
\end{marginfigure}

De la \en{Linux Foundation} (\reffig{LFlogo}) :
  \href{https://www.edx.org/professional-certificate/linuxfoundationx-open-source-software-development-linux-and-git}
  {\en{Professional Certificate in Open Source Software Development, Linux and Git}}
et en particulier la première séquence :
  \href{https://www.edx.org/course/open-source-software-development-linux-for-developers}
  {\en{Open Source Software Development: Linux for Developers}}.

\begin{description}
    \item[\en{the who}] : exemples de projets et communautés connues
    \item[\en{the what}] : définitions
    \item[\en{the why}] : les bénéfices
    \item[\en{the how}] : les licences, la conformité, conseils de collaboration, gestion de la
        diversité, développement continu et intégration
\end{description}

Les deux autres séquences s'intéressent d'un point de vue général au développement Linux et à git.

\begin{marginfigure}
    \href{https://opensource.org/}{\includegraphics[width=\marginparwidth]{osi}}
    \caption{Logo de l'\en{open source initiative}}
    \labfig{OSIlogo}
\end{marginfigure}

L'université de Calais propose actuellement un
\href{https://www.univ-littoral.fr/formation/offre-de-formation/masters/master-informatique-ingenierie-du-logiciel-libre/}{"Master
Informatique - Ingénierie du logiciel libre"}, sous la forme d'une formation en alternance.

\section*{TODO : à aller regarder}

\todo[inline]{TODO : parler de l'\href{https://opensource.org/}{OSI} (\reffig{OSIlogo}), en particulier leur
\href{https://opensource.org/osi-open-source-education}{page "\en{Education}"}.}

Peut être regarder du côté de l'\en{Oregon State University} qui, d'après \textcite{barriers-2018}, héberge un
\acrfull{OSL} hébergeant lui-même "\en{one of the largest number of Open Source projects in the world}". Ils
    font peut être des cours spécifiquement pour les logiciels libres avec un truc pareil.
