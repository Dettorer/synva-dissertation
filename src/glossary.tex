% Accronyms
\newacronym{gpl}{GPL}{GNU General Public License}
\newacronym{cc}{CC}{Creative Commons}
\newacronym{OSL}{OSL}{\en{Open Source Lab}}

% Glossary
\newglossaryentry{bug tracker}{
    name=\en{bug tracker},
    description={
        Outil se présentant généralement sous la forme d'un site web (ou d'une partie d'un site web) servant
        initialement à répertorier les \en{bugs} connus d'un projet de logiciel libre, à organiser les
        discussions autour de la résolution de ces bugs et à suivre la progression de leurs correctifs.
        L'usage de cet outil s'est aujourd'hui élargi et sert en plus à organiser et suivre de la même façon
        les discussions et changements du code liés au développement de nouvelles fonctionnalités, à la
        réorganisation du code, aux réflexions plus génériques sur la conception du projet, voir parfois le
        support utilisateur
    },
}

\newglossaryentry{commit}{
    name=\en{commit},
    description={
        Une contribution au code. Un \en{commit} est l'enregistrement d'un petit ensemble de modifications
        (suppression, ajout ou modification de une ou plusieurs lignes, dans un ou plusieurs fichiers) apporté
        au code source d'un projet
    },
}

\newglossaryentry{git}{
    name=git,
    description={
        \url{https://git-scm.com/}\\
        Logiciel de gestion de versions (dit "système de versionnement"). Git est un logiciel permettant de
        sauvegarder l'historique de développement d'un projet (ses \englpl{commit}) et d'organiser les
        contributions d'un petit ou grand nombre de personnes. Il est pensé pour mais pas limité aux logiciels
        libre, git a été conçu à l'origine pour organiser le développement du noyau de système d'exploitation
        \gls{linux}. L'historique d'un projet tel que sauvegardé par Git est communément appelé un
        "\gls{dépôt} Git"
    },
}

\newglossaryentry{dépôt}{
    name=dépôt,
    description={Voir \gls{git}},
}

\newglossaryentry{linux}{
    name=Linux,
    description={\url{https://www.kernel.org/}\\Un système d'exploitation libre},
}

\newglossaryentry{github}{
    name=GitHub,
    description={
        \url{https://github.com/}\\
        Une plateforme d'hébergement de \glspl{dépôt} \gls{git} avec \engl{bug tracker} et autres outils
        annexes, gérée par une entreprise privée
    },
}

\newglossaryentry{gitlab}{
    name=GitLab,
    description={
        \url{https://about.gitlab.com/}\\
        Un logiciel libre permettant de créer des plateformes d'hébergement de \glspl{dépôt} \gls{git} avec
        \engl{bug tracker} et autres outils annexes
    },
}

\newglossaryentry{fosdem}{
    name=FOSDEM,
    description={
        \url{https://fosdem.org}\\
        Acronyme de "\en{Free and Open Source Developers' European Meeting}". Il s'agit d'une conférence
        annuelle à l'accès gratuit hébergeant le temps d'un week-end de nombreuses présentations, ateliers et
        discussions autour du logiciel libre
    },
}

\newglossaryentry{synva}{
    name=SYNVA,
    description={
        \href{https://sfc.unistra.fr/formations/formation_-_ingenierie-de-formation_-_master-2-ingenierie-des-systemes-numeriques-virtuels-pour-lapprentissage-synva_-_2393/}
             {\nolinkurl{https://sfc.unistra.fr/[...]}}\\
        Master 2 ingénierie des SYstème Numériques Virtuels pour l'Apprentissage. Une formation dispensée par
        l'université de Strasbourg
    },
}

\newglossaryentry{epita}{
    name=EPITA,
    description={
        \url{https://www.epita.fr/}\\
        École d'ingénieurs privée, acronyme de "École pour l'Informatique et les Techniques Avancées"
    },
}

\newglossaryentry{test suite}{
    name=\en{test suite},
    description={
        Un ensemble de tests automatisés permettant de vérifier le bon fonctionnement d'un projet. Cet
        ensemble de tests s'accompagne généralement d'un outil logiciel simple permettant d'effectuer, à la
        demande, tout ou une partie des tests, puis de produire un compte rendu en résumant les résultats.
    },
}
