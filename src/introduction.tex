\chapter{Introduction}

\section{Le logiciel libre}

Au cours des quelques décennies précédentes, le numérique s'est développé pour intégrer et assister la plupart
des domaines de recherche, des secteurs industriels, ainsi que des aspects de nos vies. Son champ est
aujourd'hui très vaste et connaît en son sein de nombreuses dynamiques différentes ayant un impact sur ses
domaines d'applications, parmi lesquelles celle du logiciel libre.

En essence, les logiciels, applications, services numériques et autres programmes, sont des ensembles
d'instructions compréhensibles par un ordinateur et lui indiquant comment utiliser ses ressources afin de
réaliser un certain nombre de tâches. Pour être utilisables par un ordinateur, ces instructions doivent être
formulées en "code machine", un langage souvent représenté en binaire et extrêmement difficile à manipuler par
les humains, même les plus spécialisés, aussi bien en ce qui concerne son écriture que sa compréhension. C'est
pourquoi lors de la création d'un logiciel, les développeurs décrivent d'abord le comportement souhaité dans
un autre langage, textuel et raisonnablement maîtrisable pour un être humain spécialisé, que l'on appelle un
"langage de programmation". Cette description textuelle du logiciel est ce que l'on appelle son "code source",
il est ensuite traduit en code machine par un autre programme (généralement appelé un "compilateur") avant
d'être livré aux personnes souhaitant utiliser le logiciel.

Pour utiliser un logiciel, nous n'avons donc besoin que de son code machine, mais c'est seulement en lisant
son code \emph{source} que l'on peut vraisemblablement comprendre son fonctionnement, corriger ses éventuelles
erreurs, l'adapter à d'autres besoins, vérifier la présence de comportements malveillants, etc. La question de
rendre publiquement disponible le code source d'un logiciel est un enjeu faisant souvent intervenir plusieurs
intérêts divergents.

L'expression "logiciel libre" désigne en français un logiciel disponible publiquement et gratuitement dont le
code source est lui aussi disponible publiquement et gratuitement, sans imposer de prix à sa modification et
sa redistribution par quiconque. Cette expression désigne aussi et surtout un mouvement visant à développer la
part de logiciel libre dans le numérique, un mouvement dans lequel se développe toute une culture, des
pratiques, des fondations et des événements internationaux. L'expression équivalente anglaise, "\en{Free and
Open Source Software}" (FOSS) est souvent utilisé dans le références citées dans ce mémoire.

\section{Mon parcours personnel}

Étant issu d'une formation d'ingénieur spécialisée en informatique, mon domaine d'expertise se situe
principalement dans le numérique, dont le développement logiciel. Mon parcours m'a rapidement amené au
logiciel libre, notamment via les technologies utilisées au sein de mon école d'ingénieur, l'\gls{epita}, mais
surtout via mon implication pendant mes études dans des associations comme
Prologin\sidenote{\url{https://prologin.org}} et GConfs\sidenote{\url{https://gconfs.fr}} : l'une organisant
un concours francophone d'algorithmique ainsi que des stages d'initiation à l'informatique pour
lycéennes\sidenote{\url{https://girlscancode.fr/}}, l'autre œuvrant au partage de connaissance entre pairs en
organisant des conférences par les élèves et pour les élèves au sein de l'école. Ces deux associations ont
produit des sites web ainsi que de nombreux outils informatiques, tous sur le principe du logiciel libre, ce
qui considérablement aidé aux échanges avec leurs publiques cibles en permettant à tous de participer,
commenter et améliorer les outils développés, en plus d'accroitre la transparence des structures. Cette
implication dans le mouvement du logiciel libre m'a aussi amené à assister régulièrement au
\gls{fosdem}\sidenote{\url{https://fosdem.org}}, la conférence principale du logiciel libre en Europe, aussi
bien l'occasion de s'informer sur l'évolution des nombreuses communautés du logiciel libre qu'un prétexte pour
y retrouver de vieilles connaissances lors d'un week-end à Bruxelles.

Mon parcours s'est ensuite enrichi d'un aspect éducation lorsque j'ai rejoint pendant ma cinquième et dernière
année à l'\gls{epita} la petite équipe des "assistants", qui encadre l'enseignement de la principale matière
informatique pour les étudiants de troisième année. J'ai au sein de cette équipe écrit et évalué des sujets de
travaux pratique et encadré de nombreuses séances de travail sur ordinateur, ce qui a développé chez moi une
forte appétence pour l'enseignement. J'ai souhaité explorer cette appétence après la fin de mes études lors
d'un service civique au collège et lycée Jean Monnet de Strasbourg, où j'y ai donné des cours de soutient.
Ayant confirmé mon intérêt pour cette voie professionnelle, je suis revenu à l'\gls{epita} pour y devenir
enseignant des matières informatique, à l'occasion de l'ouverture de son nouveau campus à Strasbourg.

Enfin, des discussions avec des amis chercheurs et passionnés de méthodologie scientifique, ainsi qu'une
tentative personnelle pour me familiariser avec la recherche en éducation, m'ont poussé à entreprendre une
formation m'aidant d'une part à développer mes connaissances en éducation, et me fournissant d'autre part une
introduction à la recherche scientifique, tout en exploitant mes acquis en informatique. C'est ainsi que je me
suis inscrit au Master SYNVA et que je cherche, à travers ce mémoire, à explorer et contribuer à la recherche
touchant ces domaines de l'informatique, de l'enseignement, ainsi que de la libération de la connaissance et
de l'information.

\section{Les projets de logiciel libre dans l'enseignement de l'informatique}

La pédagogie de projet est une méthode d'enseignement qui consiste à inviter les apprenants à appliquer leurs
connaissances et à en acquérir de nouvelles au travers d'un projet plus ou moins concret et imitant plus ou
moins une situation réelle. Réalisé soit individuellement soit en groupe, les projets aboutissent généralement
à une production évaluée par l'enseignant, parfois au travers d'une présentation donnée par les apprenants.
L'efficacité de cette méthode a fait l'objet de nombreuses recherches au cours des années précédentes.
Plusieurs méta-analyses concluent que cette méthode a des effets mesurables, positifs et importants sur les
résultats académiques des apprenants en sciences sociales et en sciences naturelles
\sideparencite[-3cm]{pbl-2019, pbl-2018}.

Les projets proposés aux étudiants sont le plus souvent des projets "jouets" imaginés par les enseignants dans
le seul but de servir d'exercice. Utiliser au contraire des projets réels, qui ne cessent pas d'exister après
la fin de la séquence pédagogique, sur lesquels faire travailler les étudiants donne des résultats encore
meilleurs, mais est aussi plus difficile et incertain à encadrer pour les enseignants
\sideparencite{real-pbl-2010, real-pbl-2004}. Il est notamment très difficile pour l'enseignant de prévoir à
l'avance les problèmes que les apprenants risquent de rencontrer, c'est pourquoi d'autres efforts dans cette
direction ont tenté un compromis, consistant à créer des projets de logiciel libre aussi proches des
situations réelles que possible, mais dédiés à l'éducation \sideparencite{oss-edu-2008}. Les projets en
question ne sont cependant plus accessibles aujourd'hui, une explication possible à leur disparition étant
leur portée réduite à d'éducation.

Bien qu'elles soient plutôt rares, quelques initiatives existent aussi pour enseigner spécifiquement les
pratiques du logiciel libre dans l'éducation supérieur et la formation tout au long de la vie. Certaines sont
des initiatives venant des communautés du logiciel libre, comme le \en{Professional Certificate in Open Source
Software Development, Linux and Git}%
\sidenote{\url{https://www.edx.org/professional-certificate/linuxfoundationx-open-source-software-development-linux-and-git}}
de la \en{Linux Foundation}, et en particulier la première séquence : \en{Open Source Software Development:
Linux for Developers}%
\sidenote{\url{https://www.edx.org/course/open-source-software-development-linux-for-developers}} ; ou les
séminaires de l'\en{open source initiative}\sidenote{\url{https://opensource.org/osi-open-source-education}}.
Enfin, certaines universités proposent des cursus informatique ayant des éléments visant spécifiquement les
pratiques du logiciel libre, c'est le cas notamment de l'Université de l'État de Caroline du Nord aux
États-Unis, qui a expérimenté plusieurs façons d'inclure la contribution aux projets de logiciel libre dans
leurs cursus d'informatique \sideparencite{oss-edu-2008, oss-edu-2007} ; mais aussi de l'université de Calais
qui propose actuellement un "Master Informatique - Ingénierie du logiciel libre"%
\sidenote{\url{https://www.univ-littoral.fr/formation/offre-de-formation/masters/master-informatique-ingenierie-du-logiciel-libre/}},
sous la forme d'une formation en alternance.
