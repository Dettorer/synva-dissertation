\chapter{Introduction}

\section{Parcours personnel}

TODO, notes :

\begin{itemize}
    \item apprend les bases du code en terminale
    \item école d'ingénieur spécialisée en informatique (\gls{epita})
    \item associations Prologin et GConfs -> développement en équipe avec beaucoup de turnover (asso
        étudiante) + fort intérêt pour le partage de connaissance
    \item le \gls{fosdem} comme rendez-vous annuel
    \item rôle d'assistant pédagogique -> fort goût pour l'éducation
    \item service civique en collège/lycée pour confirmer ce goût
    \item enseignant à \gls{epita} (mentionner la première année d'administration ?)
    \item curiosité naissante pour la recherche (discussions avec des amis en thèse ou passionnés de
        méthodologie scientifique)
    \item formation \gls{synva} pour bases théoriques en enseignement, aidé de mes forces en informatique, et pour
        mettre un pied dans la porte de la recherche
    \item quelques créations et contributions à des projets de logiciel libre depuis le début de mes études
        supérieures (projets Prologin et GConfs, plugins vim, tentative citra, alacritty, nixpkgs, plugin
        thunderbird, paquets \LaTeX, \ldots)
\end{itemize}

\section{Le logiciel libre}

\todo[inline]{TODO : quelques définitions, notamment \href{https://opensource.org/osd}{open source / logiciel libre}.}

\section{Les projets de logiciel libre dans l'enseignement de l'informatique}

La pédagogie de projet est une méthode d'enseignement qui consiste à inviter les apprenants à appliquer leurs
connaissances et à en acquérir de nouvelles au travers d'un projet plus ou moins concret et imitant plus ou
moins une situation réelle. Réalisé soit individuellement soit en groupe, les projets aboutissent généralement
à une production évaluée par l'enseignant, parfois au travers d'une présentation donnée par les apprenants.
L'efficacité de cette méthode a fait l'objet de nombreuses recherches au cours des années précédentes.
Plusieurs méta-analyses concluent que cette méthode a des effets mesurables, positifs et importants sur les
résultats académiques des apprenants en sciences sociales et en sciences naturelles
\sideparencite[-3cm]{pbl-2019, pbl-2018}.

Les projets proposés aux étudiants sont le plus souvent des projets "jouets" imaginés par les enseignants dans
le seul but de servir d'exercice. Utiliser au contraire des projets réels, qui ne cessent pas d'exister après
la fin de la séquence pédagogique, sur lesquels faire travailler les étudiants donne des résultats encore
meilleurs, mais est aussi plus difficile et incertain à encadrer pour les enseignants
\sideparencite{real-pbl-2010, real-pbl-2004}. Il est notamment très difficile pour l'enseignant de prévoir à
l'avance les problèmes que les apprenants risquent de rencontrer, c'est pourquoi d'autres efforts dans cette
direction ont tenté un compromis, consistant à créer des projets de logiciel libre aussi proches des
situations réelles que possible, mais dédiés à l'éducation \sideparencite{oss-edu-2008}. Les projets en
question ne sont cependant plus accessibles aujourd'hui, une explication possible à leur disparition étant
leur portée réduite à d'éducation.

Bien qu'elles soient plutôt rares, quelques initiatives existent aussi pour enseigner spécifiquement les
pratiques du logiciel libre dans l'éducation supérieur et la formation tout au long de la vie. Certaines sont
des initiatives venant des communautés du logiciel libre, comme le \en{Professional Certificate in Open Source
Software Development, Linux and Git}%
\sidenote{\url{https://www.edx.org/professional-certificate/linuxfoundationx-open-source-software-development-linux-and-git}}
de la \en{Linux Foundation}, et en particulier la première séquence : \en{Open Source Software Development:
Linux for Developers}%
\sidenote{\url{https://www.edx.org/course/open-source-software-development-linux-for-developers}} ; ou les
séminaires de l'\en{open source initiative}\sidenote{\url{https://opensource.org/osi-open-source-education}}.
Enfin, certaines universités proposent des cursus informatique ayant des éléments visant spécifiquement les
pratiques du logiciel libre, c'est le cas notamment de l'Université de l'État de Caroline du Nord aux
États-Unis, qui a expérimenté plusieurs façons d'inclure la contribution aux projets de logiciel libre dans
leurs cursus d'informatique \sideparencite{oss-edu-2008, oss-edu-2007} ; mais aussi de l'université de Calais
qui propose actuellement un "Master Informatique - Ingénierie du logiciel libre"%
\sidenote{\url{https://www.univ-littoral.fr/formation/offre-de-formation/masters/master-informatique-ingenierie-du-logiciel-libre/}},
sous la forme d'une formation en alternance.
