\chapter{Introduction}

\section{Parcours personnel}

TODO, notes :

\begin{itemize}
    \item apprend les bases du code en terminale
    \item école d'ingénieur spécialisée en informatique (\gls{epita})
    \item associations Prologin et GConfs -> développement en équipe avec beaucoup de turnover (asso
        étudiante) + fort intérêt pour le partage de connaissance
    \item le \gls{fosdem} comme rendez-vous annuel
    \item rôle d'assistant pédagogique -> fort goût pour l'éducation
    \item service civique en collège/lycée pour confirmer ce goût
    \item enseignant à \gls{epita} (mentionner la première année d'administration ?)
    \item curiosité naissante pour la recherche (discussions avec des amis en thèse ou passionnés de
        méthodologie scientifique)
    \item formation \gls{synva} pour bases théoriques en enseignement, aidé de mes forces en informatique, et pour
        mettre un pied dans la porte de la recherche
    \item quelques créations et contributions à des projets de logiciel libre depuis le début de mes études
        supérieures (projets Prologin et GConfs, plugins vim, tentative citra, alacritty, nixpkgs, plugin
        thunderbird, paquets \LaTeX, \ldots)
\end{itemize}

\section{Le logiciel libre}

\todo[inline]{TODO : quelques définitions, notamment \href{https://opensource.org/osd}{open source / logiciel libre}.}
