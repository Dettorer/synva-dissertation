\chapter{Revue de littérature}

\section{TODO article lus mais sans catégorie}

\textcite{strategies-2012} identifient 5 grandes étapes dans les intéraction que les entreprises ont
généralement avec les logiciels libres :
\begin{description}
    \item[identification] -- évaluer la qualité d'un logiciel libre, ses garanties et les potentiels brevets
        utilisés ;
    \item[adoption] -- utilisation du logiciel ;
    \item[conformité] -- examen des licence attachées au logiciel, de leur compatibilité avec les activités de
        l'entreprise, et des contraintes qu'elles apportent dans les activités futures ;
    \item[contribution] -- partage avec la communauté.
\end{description}

\sidetextcite{strategies-2012} notent un grand besoin de formation au sein des entreprises sur le sujet des
licences, ainsi qu'un intérêt pour celle-ci à participer aux projets qu'elles utilisent.

Au sein d'un projet de logiciel libre, la communité produit généralement plus de \englpl{commit} que de
messages sur les listes de diffusion, mais il y a une corrélation entre le nombre de \englpl{commit} et le
nombre de message que chaque développeur fait \sideparencite{contribution-patterns-2010}. Cela indique qu'au
sein d'un projet de logiciel libre, bien que la production de \englpl{commit} prend plus de place, les
contributeurs participent aux échanges sur les listes de diffusion proportionnellement au nombre de
\englpl{commit} qu'ils font. Si l'on souhaite enseigner ces comportements aux étudiants, il ne faut donc pas
négliger cet aspect communication.

\section{Les barrières d'entrée}

\sidetextcite{barriers-2018} notent la difficulté des nouveaux volontaires à rejoindre une communauté de
logiciel libre, citant comme exemple extrême le projet \en{Apache Hadoop} voyant 82\% de ses nouveau
volontaire quitter le projet après leur première contribution \sideparencite{hadoop-dropout-2013} (TODO: lire
cet article en détail et comprendre les nuances). Les études que \textcite[p.~1005]{barriers-2018} citent
mentionnent deux approches différentes de la résolution de problèmes observées dans les projets informatique :
l'une consiste à d'abord rassembler le plus d'informations possibles sur le problème avant de tenter en un
deuxième temps de le résoudre (approche statistiquement plus commune chez les femmes que les hommes), l'autre
consiste à agir sur la première information ou piste prometteuse trouvée, quitte à revenir en arrière et
chercher de nouvelles informations si elle n'était pas concluante (approche statistiquement plus commune chez
les hommes que les femmes). De façon peut être plus importante pour ce mémoire, et toujours d'après les études
citées par \textcite[p.~1005]{barriers-2018}, on observe deux façons d'apprendre les fonctionnalités d'un
logiciel statistiquement préférées de façon inégales selon le genre : l'une, statistiquement plus probable
chez les femmes, consiste à les apprendre méthodiquement en suivant des processus d'apprentissage clairs.
L'autre, statistiquement plus probable chez les hommes, consiste à expérimenter à la manière d'un jeu avec ces
fonctionnalités. TODO : je dis beaucoup "les études citées par <truc>", il faudrait modifier ça pour citer les
vrais auteurs des informations dont je parle. Une des tendances trouvées par \textcite[p.~1008]{barriers-2018}
est que la majorité ($58\%$) des barrières rencontrées dans leur étude sont de nature sociale
("\en{community-oriented}") et non techniques. Concernant les aspects plus techniques, il semble que les
outils et la documentation représente la majorité des barrières rencontrées, alors même que ces éléments ont
spécifiquement pour objectif d'aider les nouvelles contributions. C'est un type de barrière qui semble
amplifié lorsque les méthodes d'apprentissage et de traitement de l'information de la personne consistent à
rassembler le plus d'information possible avant de commencer à agir. Ces méthodes étant sur-représentées chez
les femmes, ces barrières devient discriminantes selon le genre, et peuvent potentiellement participer à
expliquer la sous-représentation des femmes dans les communautés de logiciel libre. \textbf{TODO : je me suis
cette fois arrêté avant la partie 4 "\en{RELATED WORK}", page 1010}.

Il a été théorisé et souvent observé de façon générale que la diversité de genre, d'origines et d'ancienneté
au sein d'une équipe augmente sa productivité (TODO : trouver une citation générique). C'est ce que
\sidetextcite{diversity-2015} ont pu confirmer dans le cas précis des projets de logiciel libre hébergés sur
\gls{github}. Ils mettent cette observation en persective avec celle de la proportion de
femmes dans ce type d'équipe, en très forte minoritée, et concluent en suggérant que des efforts
supplémentaires d'inclusivité et de réduction des barrières d'entrée pour ces population en minorité
permettraient probablement d'augmenter la valeur globale de ce type de projets.

Liste de barrière d'entrée que rencontrent les nouveaux arrivants dans un projet de logiciel libre :

\begin{itemize}
    \item l'identification d'une tâche par laquelle commencer \sideparencite{first-task-2015}. (Note : je
        pense que ça justifie l'utilité des tags "good first issue" sur \gls{github}) ;
    \item la mise en place de l'environnement propre au projet abordé permettant de faire une première
        contribution (\sidetextcite{social-barriers-2015}, allegedly, mais je l'ai pas encore lu <- TODO)
\end{itemize}

\section{Éléments d'évaluation des formations}

TODO : trouver des éléments d'évaluation de l'efficacité / pertinence d'un cours.

\section{Méthodologies de recherches autour de ces questions}

Méthodologies de recherche employées par certains articles traitant de sujets proches de celui du présent
mémoire.

\textcite[p.~1006]{barriers-2018} analysent les réponses écrites lors d'entretiens via un codage
qualitatif suivant un modèle de \sidetextcite{barriers-2014} avec pour objectif de répondre à trois questions
de recherche : "Quels problèmes peuvent être révélé en regardant les logiciels libres au travers des outils et
de l'infrastracture ?", "Les outils et l'infrastructure participent-ils à créer des barrières d'entrée pour
les nouvelles personnes souhaitant contribuer ? Si oui, comment ?" et "Existe-t-il des barrières d'entrées
pour ces personnes qui soient biaisés sur la question du genre ? Si oui, de quelles façons sont-elles
biaisées ?".

\textcite[p.~1006]{barriers-2018} toujours utilisent le processus "\en{GenderMag}", un type de \en{Cognitive
Walkthrough} pour acquérir leurs données (TODO: se renseigner sur les détails du processus,
\textcite{barriers-2018} le survole trop vite pour que je comprenne exactement ce qui se passe).

\section{TODO potentiellemment à lire}

\begin{itemize}
    \item \sidetextcite{mining-github-2014} ;
    \item \sidetextcite{social-barriers-2015} (note: qualitatif, mais dit compléter une littérature déjà très
        quantitative, les références peut être intéressantes). Présente aussi quelques données bien aggrégées
        concernant les possibles barrières d'entrée \emph{sociales} ;
    \item des trucs sur \en{"Cognitive Walkthrough"} ? (méthode \en{science-based} d'évaluation/découverte des
        problèmes d'utilisation d'un logiciel pour les nouveaux utilisateurs)
        \sideparencite{cognitive-walkthrough-2000, cognitive-walkthrough-1994}
    \item des trucs sur "GenderMag" ? Un type de \en{Cognitive Walkthrough} spécialisé sur les problèmes liés
        au genre. \sideparencite{gendermag-2016}. Voir aussi les études citées par
        \textcite[p.~1005-1006]{barriers-2018} sur ces deux points, notamment celles traitant de leur
        efficacité.
\end{itemize}
