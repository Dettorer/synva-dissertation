\chapter{Revue de littérature}

\section{TODO article lus mais sans catégorie}

\textcite{strategies-2012} identifient 5 grandes étapes dans les intéraction que les entreprises ont
généralement avec les logiciels libres :
\begin{description}
    \item[identification] -- évaluer la qualité d'un logiciel libre, ses garanties et les potentiels brevets
        utilisés ;
    \item[adoption] -- utilisation du logiciel ;
    \item[conformité] -- examen des licence attachées au logiciel, de leur compatibilité avec les activités de
        l'entreprise, et des contraintes qu'elles apportent dans les activités futures ;
    \item[contribution] -- partage avec la communauté.
\end{description}

\sidetextcite{strategies-2012} notent un grand besoin de formation au sein des entreprises sur le sujet des
licences, ainsi qu'un intérêt pour celle-ci à participer aux projets qu'elles utilisent.

Au sein d'un projet de logiciel, la communité produit généralement plus de \englpl{commit} que de messages sur
les listes de diffusion, mais il y a une corrélation entre le nombre de \englpl{commit} et le nombre de
message que chaque développeur fait \sideparencite{contribution-patterns-2010}. Cela indique qu'au sein d'un
projet de logiciel libre, bien que la production de \englpl{commit} prend plus de place, les contributeurs
participent aux échanges sur les listes de diffusion proportionnellement au nombre de \englpl{commit} qu'ils
font. Si l'on souhaite enseigner ces comportements aux étudiants, il ne faut donc pas négliger cet aspect
communication.

\section{Les barrières d'entrée}

\sidetextcite{barriers-2018} notent la difficulté des nouveaux volontaires à rejoindre une communauté de
logiciel libre, citant comme exemple extrême le projet \en{Apache Hadoop} voyant 82\% de ses nouveau
volontaire quitter le projet après leur première contribution \sideparencite{hadoop-dropout-2013} (TODO: lire
cet article en détail et comprendre les nuances). Les études que \textcite[p.~1005]{barriers-2018} citent
mentionnent deux approches différentes de la résolution de problèmes observées dans les projets informatique :
l'une consiste à d'abord rassembler le plus d'informations possibles sur le problème avant de tenter en un
deuxième temps de le résoudre (approche statistiquement plus commune chez les femmes que les hommes), l'autre
consiste à agir sur la première information prometteuse trouvée, quitte à revenir en arrière et chercher de
nouvelles informations si la piste n'était pas concluante (approche statistiquement plus commune chez les
hommes que les femmes). De façon peut être plus importante pour ce mémoire, et toujours d'après les études
citées par \textcite[p.~1005]{barriers-2018}, on observe deux façons d'apprendre les fonctionnalités d'un
logiciel statistiquement préférées de façon inégales selon le genre : l'une, statistiquement plus probable
chez les femmes, consiste à les apprendre méthodiquement en suivant des processus d'apprentissage clairs.
L'autre, statistiquement plus probable chez les hommes, consiste à expérimenter à la manière d'un jeu avec ces
fonctionnalités. \textbf{TODO : JE M'ÉTAIS ARRÊTÉ À LA SECTION 2.3 DE \textcite{barriers-2018}, DONC DEUXIÈME
PARTIE DE LA PAGE 1006}.

Il a été théorisé et souvent observé de façon générale que la diversité de genre, d'origines et d'ancienneté
au sein d'une équipe augmente sa productivité (TODO : trouver une citation générique). C'est ce que
\sidetextcite{diversity-2015} ont pu confirmer dans le cas précis des projets de logiciel libre hébergés sur
\gls{github}. Ils mettent cette observation en persective avec celle de la proportion de
femmes dans ce type d'équipe, en très forte minoritée, et concluent en suggérant que des efforts
supplémentaires d'inclusivité et de réduction des barrières d'entrée pour ces population en minorité
permettraient probablement d'augmenter la valeur globale de ce type de projets.

Liste de barrière d'entrée que rencontrent les nouveaux arrivants dans un projet de logiciel libre :

\begin{itemize}
    \item l'identification d'une tâche par laquelle commencer \sideparencite{first-task-2015}. (Note : je
        pense que ça justifie l'utilité des tags "good first issue" sur \gls{github}) ;
    \item la mise en place de l'environnement propre au projet abordé permettant de faire une première
        contribution (\sidetextcite{social-barriers-2015}, allegedly, mais je l'ai pas encore lu <- TODO)
\end{itemize}

\section{Éléments d'évaluation des formations}

TODO : trouver des éléments d'évaluation de l'efficacité / pertinence d'un cours.

\section{TODO potentiellemment à lire}

\begin{itemize}
    \item \sidetextcite{mining-github-2014} ;
    \item \sidetextcite{social-barriers-2015} (note: qualitatif, mais dit compléter une littérature déjà très
        quantitative, les références peut être intéressantes) ;
    \item des trucs sur \en{"Cognitive Walkthrough"} ? (méthode \en{science-based} d'évaluation/découverte des
        problèmes d'utilisation d'un logiciel pour les nouveaux utilisateurs)
        \sideparencite{cognitive-walkthrough-2000, cognitive-walkthrough-1994}
    \item des trucs sur "GenderMag" ? Un type de \en{Cognitive Walkthrough} spécialisé sur les problèmes liés
        au genre. \sideparencite{gendermag-2016}
\end{itemize}
