\chapter{Revue de littérature}

\section{L'intérêt personnel dans la contribution aux projets de logiciel libre}

\todo[inline]{TODO : réordonner cette partie.}

\sidetextcite{strategies-2012} identifient cinq grandes étapes dans les interactions que les entreprises ont
généralement avec les logiciels libres :
\begin{description}
    \item[identification] -- évaluer la qualité d'un logiciel libre, ses garanties et les potentiels brevets
        utilisés ;
    \item[adoption] -- utilisation du logiciel ;
    \item[conformité] -- examen des licence attachées au logiciel, de leur compatibilité avec les activités de
        l'entreprise, et des contraintes qu'elles apportent dans les activités futures ;
    \item[contribution] -- partage avec la communauté.
\end{description}

\textcite{strategies-2012} notent un grand besoin de formation au sein des entreprises sur le sujet des
licences, ainsi qu'un intérêt pour celle-ci à participer aux projets qu'elles utilisent. D'autres auteurs
avancent qu'avoir produit des contributions à des projets de logiciel libre envoi un signal positif à ses
pairs, augmentant les chances de succès professionnel dans le monde de l'informatique
\sideparencite[][p. 218]{oss-economics-2002}.

Au sein d'un projet de logiciel libre, la communauté produit généralement plus de \englpl{commit} que de
messages sur les listes de diffusion, mais il y a une corrélation entre le nombre de \englpl{commit} et le
nombre de messages que chaque développeur écrit \sideparencite{contribution-patterns-2010}. Cela indique qu'au
sein d'un projet de logiciel libre, bien que la production de \englpl{commit} prend plus de place, les
contributeurs participent aux échanges sur les listes de diffusion proportionnellement au nombre de
\englpl{commit} qu'ils font. Si l'on souhaite enseigner ces comportements aux étudiants, il ne faut donc pas
négliger cet aspect communication.

\section{Pédagogie de projet}

\todo[inline]{TODO : supprimer cette partie ou la mentionner en intro comme application possible de ma
problématique}

La pédagogie de projet est une méthode d'enseignement qui consiste à inviter les apprenants à appliquer leurs
connaissances et à en acquérir de nouvelles au travers d'un projet plus ou moins concret et imitant plus ou
moins une situation réelle. Réalisé soit individuellement soit en groupe, les projets aboutissent généralement
à une production évaluée par l'enseignant, parfois au travers d'une présentation donnée par les apprenants.
L'efficacité de cette méthode a fait l'objet de nombreuses recherches au cours des années précédentes.
Plusieurs méta-analyses concluent que cette méthode a des effets mesurables, positifs et importants sur les
résultats académiques des apprenants en sciences sociales et en sciences naturelles
\sideparencite[-3cm]{pbl-2019, pbl-2018}.

Les projets proposés aux étudiants sont le plus souvent des projets "jouets" imaginés par les enseignants dans
le seul but de servir d'exercice. Utiliser au contraire des projets réels, qui ne cessent pas d'exister après
la fin de la séquence pédagogique, sur lesquels faire travailler les étudiants donne des résultats encore
meilleurs, mais est aussi plus difficile et incertain à encadrer pour les enseignants
\sideparencite{real-pbl-2010, real-pbl-2004}. Il est notamment très difficile pour l'enseignant de prévoir à
l'avance les problèmes que les apprenants risquent de rencontrer.

\section{Les projets de logiciel libre dans l'enseignement de l'informatique}

\todo[inline]{TODO : supprimer cette partie ou la mentionner en intro comme application possible de ma
problématique}

Lors de leurs tentatives d'inclusion des projets de logiciel libre dans leurs cours d'informatique (voir
chapitre \ref{chap:existing_teaching}, page \pageref{teaching:ncsu}), \sidetextcite{oss-edu-2008} ont
initialement cherché à travailler avec un projet existant, mais n'en ont trouvé aucun satisfaisant leurs
critères. Leur solution a alors été de créer eux-mêmes quelques projets de logiciel libre à destination de
leurs étudiants, avec pour but d'imiter autant que possible le contexte d'un projet réel. Leur démarche était
de créer une banque de projets de logiciel libre utilisables spécifiquement en éducation, mais ceux-ci
n'existent malheureusement plus aujourd'hui.

\section{Les nouveaux contributeurs dans le logiciel libre}

\subsection{Les barrières d'entrée}

\sidetextcite{barriers-2018} notent la difficulté des nouveaux volontaires à rejoindre une communauté de
logiciel libre, citant comme exemple extrême le projet \en{Apache Hadoop} voyant 82\% de ses nouveaux
volontaires quitter le projet après leur première contribution \sideparencite{hadoop-dropout-2013}. Les
études, que \textcite[p.~1005]{barriers-2018} citent, mentionnent deux approches différentes de la résolution
de problèmes observées dans les projets informatique : l'une consiste à d'abord rassembler le plus
d'informations possibles sur le problème avant de tenter en un deuxième temps de le résoudre (approche
statistiquement plus commune chez les femmes que les hommes), l'autre consiste à agir sur la première
information ou piste prometteuse trouvée, quitte à revenir en arrière et chercher de nouvelles informations si
elles n'étaient pas concluantes (approche statistiquement plus commune chez les hommes que les femmes). De
façon peut être plus importante pour ce mémoire, et toujours d'après les études citées par
\textcite[p.~1005]{barriers-2018}, on observe deux façons d'apprendre les fonctionnalités d'un logiciel
statistiquement préférées de façon inégales selon le genre : l'une, statistiquement plus probable chez les
femmes, consiste à les apprendre méthodiquement en suivant des processus d'apprentissage clairs. L'autre,
statistiquement plus probable chez les hommes, consiste à expérimenter à la manière d'un jeu avec ces
fonctionnalités. \todo{Je dis beaucoup "les études citées par <truc>", il faudrait modifier ça pour citer les
vrais auteurs des informations dont je parle.} Une des tendances trouvées par
\textcite[p.~1008]{barriers-2018} est que la majorité ($58\%$) des barrières rencontrées dans leur étude sont
de nature sociale ("\en{community-oriented}") et non techniques. Concernant les aspects plus techniques, il
semble que les outils et la documentation représentent la majorité des barrières rencontrées, alors même que
ces éléments ont spécifiquement pour objectif d'aider les nouvelles contributions. C'est un type de barrière
qui semble amplifié lorsque les méthodes d'apprentissage et de traitement de l'information de la personne
consistent à rassembler le plus d'information possible avant de commencer à agir. Ces méthodes étant
sur-représentées chez les femmes, ces barrières devient discriminantes selon le genre, et peuvent
potentiellement participer à expliquer la sous-représentation des femmes dans les communautés de logiciel
libre.

Plus précisément concernant \sidetextcite{hadoop-dropout-2013}, les facteurs susceptibles de provoquer un
abandon des nouveaux contributeurs semblent être les réponses inadéquates proposées aux questions de ces
nouveaux contributeurs, notamment lorsqu'un autre nouveau contributeur répond à la place d'un membre
expérimenté du projet. L'absence de réponse, en revanche, ne semble avoir que très peu d'influence.

Il a été théorisé et souvent observé de façon générale que la diversité de genre, d'origines et d'ancienneté
au sein d'une équipe augmente sa productivité \todo{Trouver une citation.}. C'est ce que
\sidetextcite{diversity-2015} ont pu confirmer dans le cas précis des projets de logiciel libre hébergés sur
\gls{github}. Ils mettent cette observation en perspective avec celle de la proportion de
femmes dans ce type d'équipe, en très forte minorité, et concluent en suggérant que des efforts
supplémentaires d'inclusivité et de réduction des barrières d'entrée pour ces population en minorité
permettraient probablement d'augmenter la valeur globale de ce type de projets.

D'après les articles que j'ai pu lire, les plus importantes barrières d'entrée que rencontrent les nouveaux
arrivants dans un projet de logiciel libre sont :

\begin{itemize}
    \item le manque d'interactions sociales avec les membres du projet \sideparencite{barriers-meta-2015} ;
    \item le manque de réponse (dans un temps raisonnable) à leurs requêtes \parencite{barriers-meta-2015} ;
    \item les connaissances techniques initiales \parencite{barriers-meta-2015} ;
    \item l'identification d'une tâche par laquelle commencer \sideparencite{first-task-2015} ;
        \todo[color=green]{Note : je pense que ça justifie l'utilité des tags "good first issue" sur
        \gls{github}}
    \item la mise en place de l'environnement propre au projet abordé permettant de faire une première
        contribution \sideparencite{newcomers-accessibility-2016}.
\end{itemize}

\subsection{Mesure de l'accessibilité d'un projet pour les nouveaux contributeurs}

Ces modèles de barrières d'entrée permettent d'identifier les freins que rencontrent les nouveaux
contributeurs dans un projet de logiciel libre. Un autre point de vue sur la question consiste à mesurer
\emph{a posteriori} la capacité d'un projet à accueillir les nouveaux contributeurs et à leur permettre de
mener leurs contributions à terme.

Une première approche à cela, qualitative, est de proposer à un ensemble maîtrisé d'étudiants d'essayer de
contribuer à un projet et de leur soumettre un questionnaire avant et/ou après l'expérience, puis
d'éventuellement compléter avec des entretiens
\parencites{newcomers-accessibility-2016}{newcomers-onboarding-2018}[voir aussi][]{newcomers-adaptation-2005}.

Une autre approche, peut être plus adaptée aux recherches quantitatives, consiste à compter automatiquement le
nombre de contributeurs accumulés depuis le début de la vie du projet et jusqu'à différents points dans le
temps, afin d'étudier la progression de ce nombre. C'est ce que font par exemple
\sidetextcite{contributor-count-2006}, mais sans préciser la méthode de comptage plus que "en utilisant
plusieurs outils d'édition de texte" (p. 116).

\sidetextcite{signals-2019} suggèrent qu'une mesure possible de l'accessibilité pour les nouveaux
contributeurs ("\en{newcomers openness}" dans l'article) est le pourcentage de \englpl{pull request} créées
par des contributeurs externes. Ils définissent les "contributeurs externes" par opposition aux contributeurs
principaux (les \en{core contributors}) qu'ils identifient, eux, comme étant les personnes auteurs de plus de
5\% des \englpl{commit} du projet.

\todo[inline]{%
    TODO : lire tout \textcite{signals-2019}, c'est long mais c'est en plein dans ce que je veux faire.%
}

\todo[inline]{TODO : à étoffer}

\section{L'analyse automatique ("minage") de projets de logiciel libre}

Sur le minage de façon générale :

\begin{itemize}
    \item \url{https://ieeexplore.ieee.org/document/4659248}
    \item \url{https://onlinelibrary.wiley.com/doi/10.1002/smr.344}
    \item \url{https://hal.archives-ouvertes.fr/hal-02158292}
    \item \sidetextcite{mining-github-2014}
\end{itemize}

\subsection{L'archive de logiciels de Software Heritage}

\todo[inline]{%
    TODO : parler de la recherche qu'il y a autour de la construction du graph (et les motivations derrières,
    par exemple), et essayer de trouver des expériences qui s'en servent%
}

\begin{itemize}
    \item \sidetextcite{swh-graph-2020} : minage à grande échelle via l'API swh-graph de Software Heritage,
        avec un exemple de mesure sur le graph compressé pour montrer qu'on peut miner des vraies choses en un
        temps raisonnable sur la totalité du corpus.
    \item \url{https://dl.acm.org/doi/10.1145/3183558}, \url{https://hal.archives-ouvertes.fr/hal-01590958/}
        et \url{https://ieeexplore.ieee.org/abstract/document/8816748} pour justifier l'usage de l'archive de
        Software Heritage comme meilleur approximation existante de l'entiereté du code source développé
        publiquement.
\end{itemize}

\section{Méthodologies de recherches autour de ces questions}

Méthodologies de recherche employées par certains articles traitant de sujets proches de celui du présent
mémoire.

\textcite[p.~1006]{barriers-2018} analysent les réponses écrites lors d'entretiens via un codage
qualitatif suivant un modèle de \sidetextcite{barriers-2014} avec pour objectif de répondre à trois questions
de recherche : "Quels problèmes peuvent être révélé en regardant les logiciels libres au travers des outils et
de l'infrastructure ?", "Les outils et l'infrastructure participent-ils à créer des barrières d'entrée pour
les nouvelles personnes souhaitant contribuer ? Si oui, comment ?" et "Existe-t-il des barrières d'entrées
pour ces personnes qui soient biaisés sur la question du genre ? Si oui, de quelles façons sont-elles
biaisées ?".

\textcite[p.~1006]{barriers-2018} toujours utilisent le processus "\en{GenderMag}", un type de \en{Cognitive
Walkthrough} pour acquérir leurs données.
\todo{%
    TODO : décrire un peu plus le processus (ou ne pas en parler si ce n'est pas pertinent, vu que ma
    méthodologie sera complètement différente de ça).%
}

\subsection{Sélection des projets étudiés}

Dans une méta-analyse concernant l'étude des barrières d'entrée, \sidetextcite{barriers-meta-2015} remarquent
un biais courant dans la sélection des projets de logiciel libre retenus dans les études (p. 83) : les auteurs
ont tendance à plutôt étudier les projets importants, matures et populaires, car ils sont plus susceptible
d'apporter un volume important de données. Bien que les résultats semblent cohérent avec les plus rares études
sélectionnant de petits projets, ils invitent la communauté scientifique à s'y intéresser plus en profondeur
ou à plus varier ses échantillons.

\todo[inline]{TODO : quelques articles potentiellement à lire}

\begin{itemize}
    \item \sidetextcite{social-barriers-2015} (note: qualitatif, mais dit compléter une littérature déjà très
        quantitative, les références peut être intéressantes). Présente aussi quelques données bien agrégées
        concernant les possibles barrières d'entrée \emph{sociales} ;
    \item des trucs sur \en{"Cognitive Walkthrough"} ? (méthode \en{science-based} d'évaluation/découverte des
        problèmes d'utilisation d'un logiciel pour les nouveaux utilisateurs)
        \sideparencite{cognitive-walkthrough-2000, cognitive-walkthrough-1994}
    \item des trucs sur "GenderMag" ? Un type de \en{Cognitive Walkthrough} spécialisé sur les problèmes liés
        au genre. \sideparencite{gendermag-2016}. Voir aussi les études citées par
        \textcite[p.~1005-1006]{barriers-2018} sur ces deux points, notamment celles traitant de leur
        efficacité.
\end{itemize}

Spécifiquement autour de l'idée d'identifier les bons projets de logiciel libre vers lesquels diriger des
étudiants qui découvrent le milieu :

\begin{itemize}
    \item \sidetextcite{mining-github-2014}.
    \item \sidetextcite{signals-2019}, choisir des projets typiquement attirants peut aider.
    \item \sidetextcite{code-of-conduct-2017}, n'a pas l'air très concluant, à regarder rapidement.
    \item "Terell et al. [55]" dans \textcite{barriers-2018}, cité page 1011, sur comment miner github.
\end{itemize}

Quand \sidetextcite{mining-github-2014} analysent les \en{pull requests} sur github, ils ne retiennent que les
projets ayant au moins deux \en{commiters}, ce qu'ils considèrent comme des projets "vraiment
collaboratifs".\todo{développer et bien intégrer au reste de la partie}
