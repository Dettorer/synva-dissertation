\chapter{Revue de littérature}

\sidetextcite{strategies-2012} identifient 5 grandes étapes dans les intéraction que les entreprises ont
généralement avec les logiciels libres :
\begin{description}
    \item[identification] : évaluer la qualité d'un logiciel libre, ses garanties et les potentiels brevets
        utilisés ;
    \item[adoption] : utilisation du logiciel ;
    \item[conformité] : examen des licence attachées au logiciel, de leur compatibilité avec les activités de
        l'entreprise, et des contraintes qu'elles apportent dans les activités futures ;
    \item[contribution] : partage avec la communauté.
\end{description}

\textcite{strategies-2012} notent un grand besoin de formation au sein des entreprises sur le sujet des
licences, ainsi qu'un intérêt pour celle-ci à participer aux projets qu'elles utilisent.

Au sein d'un projet de logiciel, la communité produit généralement plus de \englpl{commit} que de messages sur
les listes de diffusion, mais il y a une corrélation entre le nombre de \englpl{commit} et le nombre de
message que chaque développeur fait \sideparencite{contribution-patterns-2010}. Cela indique qu'au sein d'un
projet de logiciel libre, bien que la production de \englpl{commit} prend plus de place, les contributeurs
participent aux échanges sur les listes de diffusion proportionnellement au nombre de \englpl{commit} qu'ils
font. Si l'on souhaite enseigner ces comportements aux étudiants, il ne faut donc pas négliger cet aspect
communication.

\sidetextcite{barriers-2018} notent la difficulté des nouveaux volontaires à rejoindre une communauté de
logiciel libre, citant comme exemple extrême le projet \en{Apache Hadoop} voyant 82\% de ses nouveau
volontaire quitter le projet après leur première contribution \sideparencite{hadoop-dropout-2013} (TODO: lire
cet article en détail et comprendre les nuances).
