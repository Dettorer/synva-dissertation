\chapter{Méthodologie}

\section{Hypothèses}

TODO : numéroter les hypothèses (je suis sûr qu'un paquet latex saurait le faire automatiquement !).

\begin{itemize}
    \item les projets possédant un fichier "\en{README}" sont plus accessibles que ceux n'en ayant pas ;
    \item plus le fichier "\en{README}" est grand (quand il existe), plus le projet est accessible ;
\end{itemize}

\section{Mesure de l'accessibilité choisie}

TODO : méthode choisie pour mesurer à quel point un projet est accessibilité pour les nouveaux contributeurs.

\section{Constitution de l'échantillon}

TODO : quels critères de sélection des projets sur Software Heritage (ou aussi github/gitlab ?), critères
d'exclusion.

\section*{TODO : penser à mettre le code en annexe}
