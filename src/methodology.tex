\chapter{Méthodologie}

\newtheorem{hypo}{Hypothèse}
\newcommand{\newhyp}[2]{%
    \begin{restatable}{hypothesis}{#1}
        \label{hyp:#1}#2
    \end{restatable}%
}

\section{Hypothèses}

\newhyp{contributionguidelines}{%
    les projets possédant des instructions de contribution (fichier type "\en{CONTRIBUTING.md}" ou section
    type "\en{Contributing}" dans le \en{README}) sont plus accessibles pour les nouveaux contributeurs que
    ceux n'en ayant pas \sideparencite[][voir][p.~11]{signals-2019}.%
}

\newhyp{recentcontributorcount}{%
    le nombre de contributeurs uniques récents (du premier janvier 2019 au premier juin 2019) d'un projet
    influe sur son accessibilité pour les nouveaux contributeurs \parencite[voir][p.~12-13,16]{signals-2019}.%
}

\newhyp{recentcommitcount}{%
    le nombre de \englpl{commit} récents (du premier janvier 2019 au premier juin 2019) au sein d'un projet
    influe sur son accessibilité pour les nouveaux contributeurs \parencite[voir][p.~13,16]{signals-2019}.
}

\section{Mesure de l'accessibilité choisie}

La variable mesurée pour représenter l'accessibilité d'un projet pour les nouveaux contributeurs est le nombre
de contributeurs ayant fait leur toute première contribution au projet entre le premier juin 2019 et le
premier septembre 2019. (voir section \ref{sec:accessibility-measure} p.~\pageref{sec:accessibility-measure},
ainsi que \textcite[][p.~13,16]{signals-2019})

\section{Constitution de l'échantillon}

L'échantillon de départ est constitué de la totalité des projets archivés dans
le graph de Software Heritage. Plusieurs critères d'exclusion ont ensuite été
appliqués :

\begin{itemize}
    \item quand deux projets ou plus ont des \englpl{commit} en commun, seul
        celui qui a reçu le plus d'activité (mesuré en regardant le nombre de
        \englpl{commit} entre le premier et le dernier) a été retenu ;
    \item les projets n'ayant enregistré aucune activité (aucun commit) entre le
        1er juin 2019 et le 1er septembre 2019 n'ont pas été retenus ;
\end{itemize}

\section{Collecte des données}

\todo[inline]{%
    TODO : parcours du graph de Software Heritage avec l'API graph
    (\url{https://docs.softwareheritage.org/devel/swh-graph/index.html}) soit sur un sous-graph compressé
    (\url{https://docs.softwareheritage.org/devel/swh-dataset/graph/dataset.html}) qui tient sur mon
    ordinateur personnel, soit sur le graph complet si je trouve quelqu'un pour me donner un peu de temps de
    calcul sur un gros serveur.%
}

\todo[inline]{TODO : penser à mettre le code en annexe}

\section{Traitement des données}

\todo[inline]{TODO}

\todo[inline]{%
    TODO bonus : faire un "\en{replication package}" avec mon code, comme le font \textcite{swh-graph-2020}
    ici : \url{https://zenodo.org/record/3574459} ?%
}
