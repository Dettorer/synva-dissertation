\chapter{Méthodologie}

\section{Hypothèses}

\todo[inline]{TODO : numéroter les hypothèses (je suis sûr qu'un paquet latex saurait le faire
automatiquement !).}

\begin{itemize}
    \item les projets possédant un fichier "\en{README}" sont plus accessibles pour les nouveaux contributeurs
        que ceux n'en ayant pas ;
    \item plus le fichier "\en{README}" est grand (quand il existe), plus le projet est accessible pour les
        nouveaux contributeurs ;
    \item les projets possédant une \engl{test suite} sont plus accessibles pour les nouveaux contributeurs que
        ceux n'en ayant pas ;
    \item le langage principal d'un projet influe sur son accessibilité pour les nouveaux contributeurs ;
    \item la taille d'un projet (en ligne de code) influe sur son accessibilité pour les nouveaux
        contributeurs ;
    \item le nombre de contributeurs réguliers influe sur son accessibilité pour les nouveaux contributeurs ;
\end{itemize}

\section{Mesure de l'accessibilité choisie}

\todo[inline]{TODO : méthode choisie pour mesurer à quel point un projet est accessible pour les nouveaux
contributeurs.}

\todo[inline, color=yellow]{Idée : regarder chaque mois (ou trimestre ? ou année ?) le pourcentage de contributeurs (auteurs
uniques de \engl{commit}) qui contribue dans ce projet pour la première fois.}

\section{Constitution de l'échantillon}

\todo[inline]{TODO : quels critères de sélection des projets sur Software Heritage (ou aussi github/gitlab ?),
critères d'exclusion.}

\section*{TODO : penser à mettre le code en annexe}
