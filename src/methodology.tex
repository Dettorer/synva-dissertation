\chapter{Méthodologie}

\newtheorem{hypo}{Hypothèse}
\newcommand{\newhyp}[2]{%
    \begin{restatable}{hypothesis}{#1}
        \label{hyp:#1}#2
    \end{restatable}%
}

\todo[inline, color=yellow]{%
    Mon intuition est de trouver une ou plusieurs métriques dans les projets de logiciel libre pouvant être
    prédictives de leur tendance à bien accueillir les nouveaux contributeurs (que l'on pourrait par exemple
    mesurer en regardant la régularité avec laquelle de nouveaux noms apparaissent dans le git log). De tels
    facteurs pourraient servir lors de la préparation de cours consistant à participer à un projet de logiciel
    libre, l'enseignant·e pourrait identifier les projets les plus susceptible de convenablement accueillir
    ses étudiant·es.%
}

\section{Hypothèses}

\newhyp{contributing.md}{%
    les projets possédant un fichier "\en{CONTRIBUTING.md}" sont plus accessibles pour les nouveaux
    contributeurs que ceux n'en ayant pas \sideparencite[][voir][p.~11]{signals-2019}.%
}

\newhyp{readmeexists}{%
    les projets possédant un fichier "\en{README}" sont plus accessibles pour les nouveaux contributeurs que
    ceux n'en ayant pas.%
}

\newhyp{readmelength}{%
    plus le fichier "\en{README}" est grand (quand il existe), plus le projet est accessible pour les nouveaux
    contributeurs.%
}

\newhyp{testsuite}{%
    les projets possédant une \engl{test suite} sont plus accessibles pour les nouveaux contributeurs que ceux
    n'en ayant pas.%
}
\todo[color=red!80!white]{%
    Sans doute trop dur à vérifier automatiquement, surtout si je ne me limite pas à un seul langage%
}

\newhyp{mainlanguage}{%
    le langage principal d'un projet influe sur son accessibilité pour les nouveaux contributeurs.%
}

\newhyp{codelength}{%
    la taille d'un projet (en ligne de code) influe sur son accessibilité pour les nouveaux contributeurs.%
}

\newhyp{contributorcount}{%
    le nombre de contributeurs réguliers d'un projet influe sur son accessibilité pour les nouveaux
    contributeurs.%
}

\section{Mesure de l'accessibilité choisie}

\todo[inline]{%
    TODO : méthode choisie pour mesurer à quel point un projet est accessible pour les nouveaux
    contributeurs.%
}

\todo[inline, color=yellow]{%
    Idée : regarder chaque mois (ou trimestre ? ou année ?) le pourcentage de contributeurs (auteurs uniques
    de \engl{commit}) qui contribue dans ce projet pour la première fois. Malheureusement pour l'instant je
    n'ai vu personne essayer de mesurer ce genre de chose donc il n'y a que mon intuition pour me dire que
    c'est une mesure correcte de l'accessibilité pour les nouveaux contributeurs.%
}

\section{Constitution de l'échantillon}

\todo[inline]{%
    TODO : quels critères de sélection des projets sur Software Heritage (ou aussi github/gitlab ?), critères
    d'exclusion.
}

\section{Collecte des données}

\todo[inline]{%
    TODO : parcours du graph de Software Heritage avec l'API graph
    (\url{https://docs.softwareheritage.org/devel/swh-graph/index.html}) soit sur un sous-graph compressé
    (\url{https://docs.softwareheritage.org/devel/swh-dataset/graph/dataset.html}) qui tient sur mon
    ordinateur personnel, soit sur le graph complet si je trouve quelqu'un pour me donner un peu de temps de
    calcul sur un gros serveur.%
}

\todo[inline]{TODO : penser à mettre le code en annexe}

\section{Traitement des données}

\todo[inline]{TODO}

\todo[inline]{%
    TODO bonus : faire un "\en{replication package}" avec mon code, comme le font \textcite{swh-graph-2020}
    ici : \url{https://zenodo.org/record/3574459} ?%
}
